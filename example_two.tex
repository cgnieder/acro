% arara: pdflatex
% arara: pdflatex
% this file is an adapted version of acrotest.tex shipped out with the `acronym'
% package
\documentclass{article}
\usepackage[colorlinks]{hyperref}
\usepackage{acro}
\acsetup{sort,page-ref=comma,extra-style=paren,hyperref}

\DeclareAcronym{CDMA}{CDMA}{Code Division Multiple Access}
\DeclareAcronym{GSM}{GSM}{Global System for Mobile communication}
\DeclareAcronym{NA}{\ensuremath{N_{\mathrm A}}}{Number of Avogadro}{see \S\ref{Chem}}
\DeclareAcronym{NAD+}{NAD\textsuperscript{+}}{Nicotinamide Adenine Dinucleotide}
\DeclareAcronym{NUA}{NUA}{Not Used Acronym}
\DeclareAcronym{TDMA}{TDMA}{Time Division Multiple Access}
\DeclareAcronym{UA}{UA}{Used Acronym}
\DeclareAcronym{lox}{\ensuremath{LOX}}{Liquid Oxygen}%
\DeclareAcronym{lh2}{\ensuremath{LH_2}}{Liquid Hydrogen}%
\DeclareAcronym{IC}{IC}{Integrated Circuit}%
\DeclareAcronym*{BUT}{BUT}{Block Under Test,Blocks Under Test}%
\begin{document}

\section{Intro}
In the early nineties, \acs{GSM} was deployed in many European
countries. \ac{GSM} offered for the first time international
roaming for mobile subscribers. The \acs{GSM}'s use of \ac{TDMA} as
its communication standard was debated at length. And every now
and then there are big discussion whether \ac{CDMA} should have
been chosen over \ac{TDMA}.

\section{Furthermore}
\acreset
The reader could have forgotten all the nice acronyms, so we repeat the
meaning again.

If you want to know more about \acf{GSM}, \acf{TDMA}, \acf{CDMA}
and other acronyms, just read a book about mobile communication. Just
to mention it: There is another \ac{UA}, just for testing purposes!

\begin{figure}[h]
Figure
\caption{A float also admits references like \ac{GSM} or \acf{CDMA}.}
\end{figure}

\subsection{Some chemistry and physics}
\label{Chem}
\ac{NAD+} is a major electron acceptor in the oxidation
of fuel molecules. The reactive part of \ac{NAD+} is its nictinamide
ring, a pyridine derivate.

One mol consists of \acs{NA} atoms or molecules. There is a relation
between the constant of Boltzmann and the \acl{NA}:
\begin{equation}
  k = R/\acs{NA}
\end{equation}

\acl{lox}/\acl{lh2} (\acs{lox}/\acs{lh2})

\subsection{Some testing fundamentals}
When testing \acp{IC}, one typically wants to identify functional
blocks to be tested separately. The latter are commonly indicated as
\acp{BUT}. To test a \ac{BUT} requires defining a testing strategy\dots

\printacronyms

\end{document}