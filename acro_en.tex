% arara: pdflatex
% !arara: biber
% !arara: pdflatex
% !arara: pdflatex
% arara: pdflatex
% --------------------------------------------------------------------------
% the ACRO package
% 
%   Typeset Acronyms
% 
% --------------------------------------------------------------------------
% Clemens Niederberger
% Web:    https://bitbucket.org/cgnieder/acro/
% E-Mail: contact@mychemistry.eu
% --------------------------------------------------------------------------
% Copyright 2011-2015 Clemens Niederberger
% 
% This work may be distributed and/or modified under the
% conditions of the LaTeX Project Public License, either version 1.3
% of this license or (at your option) any later version.
% The latest version of this license is in
%   http://www.latex-project.org/lppl.txt
% and version 1.3 or later is part of all distributions of LaTeX
% version 2005/12/01 or later.
% 
% This work has the LPPL maintenance status `maintained'.
% 
% The Current Maintainer of this work is Clemens Niederberger.
% --------------------------------------------------------------------------
% The acro package consists of the files
%  - acro.sty, acro0.def, acro1.def, acro_en.tex, acro_en.pdf, README
% --------------------------------------------------------------------------
% If you have any ideas, questions, suggestions or bugs to report, please
% feel free to contact me.
% --------------------------------------------------------------------------
%
% if you want to compile this documentation you'll need the document class
% `cnpkgdoc' which you can get here:
%    https://bitbucket.org/cgnieder/cnpkgdoc/
% the class is licensed LPPL 1.3 or later
%

\documentclass[load-preamble+,scrartcl={DIV10}]{cnltx-doc}
\usepackage[utf8]{inputenc}
\usepackage[single,macros,accsupp,index]{acro}
\setcnltx{
  package  = {acro} ,
  authors  = Clemens Niederberger ,
  email    = contact@mychemistry.eu ,
  url      = https://bitbucket.org/cgnieder/acro/ ,
  abstract = {%
    \acro\ not only allows you to create acronyms in a simple way but also lets
    you add them to different classes of acronyms. Lists can be created of
    separate classes wherever you want the list to appear.\par
    \acro\ also provides an option \option{single} which ignores acronyms that are
    used only once in the whole document.\par
    As an experimental feature \acro\ also offers the option \option{sort} which
    automatically sorts the list created by \cs{printacronyms}.%
  } ,
  add-cmds = {
    ac,aca,acap,acf,acflike,acfp,acfplike,acl,aclp,acp,acs,acsp,
    acreset,acresetall,acsetup,
    DeclareAcronym,DeclareAcronymCitation,DeclareAcronymFormat,
    DeclareAcronymPDFString,
    iac,Iac,iaca,Iaca,iacs,Iacs,iacl,Iacl,iacf,Iacf,iacflike,Iacflike,
    printacronyms
  } ,
  add-silent-cmds = {
    addcolon,nato,newlist,ny,setlist
  } ,
  index-setup = { level = \section , headers={\indexname}{\indexname} }
}

\acsetup{hyperref}

\defbibheading{bibliography}{\section{References}}

\usepackage{csquotes}

\usepackage[biblatex]{embrac}[2012/06/29]
\ChangeEmph{[}[,.02em]{]}[.055em,-.08em]
\ChangeEmph{(}[-.01em,.04em]{)}[.04em,-.05em]

\usepackage{filecontents}

\addbibresource{\jobname.bib}
\begin{filecontents}{\jobname.bib}
@online{wikipedia,
  author   = {Wikipedia},
  title    = {Acronym and initialism},
  urldate  = {2012-06-21},
  url      = {http://en.wikipedia.org/wiki/Acronyms},
  year     = {2012}
}
@online{NewYork,
  author   = {Wikipedia},
  title    = {New York City},
  urldate  = {2012-09-27},
  url      = {http://en.wikipedia.org/wiki/New_York_City},
  year     = {2012}
}
\end{filecontents}

% additional packages:
\usepackage{longtable,array,booktabs,enumitem,amssymb}

\newcommand*\wikipedia{%\libertineGlyph{W.alt}\kern-.055em
\textsc{Wikipedia}}
\newcommand*\h[1]{\textcolor{cnltx}{\textbf{#1}}}

% declare acronyms
\DeclareAcronym{cd}
  {
    short        = cd ,
    long         = Compact Disc ,
    short-format = \scshape
  }
\let\ctan\relax
\DeclareAcronym{ctan}
  {
    short     = ctan ,
    long      = Comprehensive \TeX\ Archive Network ,
    format    = \scshape ,
    pdfstring = CTAN ,
    accsupp   = CTAN
  }
\def\ctan{\acs{ctan}}
\DeclareAcronym{ecu}
  {
    short   = ECU ,
    long    = Steuerger\"at ,
    foreign = Electronic Control Unit
  }
\DeclareAcronym{id}
  {
    short        = id ,
    long         = identification string ,
    short-format = \scshape
  }
\DeclareAcronym{jpg}
  {
    short = JPEG ,
    alt   = JPG ,
    long  = Joint Photographic Experts Group
  }
\DeclareAcronym{la}
  {
    short        = LA ,
    short-plural = ,
    long         = Los Angeles,
    long-plural  = ,
    class        = city
  }
\let\lppl\relax
\DeclareAcronym{lppl}
  {
    short     = lppl ,
    long      = \LaTeX\ Project Public License ,
    format    = \scshape ,
    pdfstring = LPPL ,
    accsupp   = LPPL ,
    index-cmd = \csname @gobble\endcsname
  }
\def\lppl{\acs{lppl}}
\DeclareAcronym{MP}
  {
    short = MP ,
    long  = Member of Parliament ,
    long-plural-form = Members of Parliament
  }
\DeclareAcronym{nato}
  {
    short        = nato ,
    long         = North Atlantic Treaty Organization ,
    extra        = \emph{deutsch}: Organisation des Nordatlantikvertrags ,
    short-format = \scshape
  }
\DeclareAcronym{ny}
  {
    short        = NY ,
    short-plural = ,
    long         = New York ,
    long-plural  = ,
    class        = city ,
    cite         = NewYork
  }
\DeclareAcronym{ot}
  {
    short        = ot ,
    long         = Other Test ,
    short-format = \scshape
  }
\DeclareAcronym{pdf}
  {
    short     = pdf ,
    long      = Portable Document Format ,
    format    = \scshape ,
    pdfstring = PDF ,
    accsupp   = PDF
  }
\DeclareAcronym{sw}
  {
    short       = SW ,
    long        = Sammelwerk ,
    long-plural = e
  }
\DeclareAcronym{test}
  {
    short = ST ,
    long  = Some Test
  }
\DeclareAcronym{tex.sx}
  {
    short = \TeX.sx ,
    sort  = TeX.sx ,
    long  = \TeX{} StackExchange
  }
\DeclareAcronym{ufo}{
   short           = UFO ,
   long            = unidentified flying object ,
   long-indefinite = an
}


\makeatletter
\protected\def\@versionstar{\raisebox{-.25em}{*}}
\newcommand\versionstar{\texorpdfstring{\@versionstar}{*}}
\makeatother

\newcommand*\latin{\textit}

\begin{document}
\section{Licence and Requirements}
\license

\acro\ loads and needs the following packages:
\pkg{expl3}\footnote{\CTANurl{l3kernel}}, \pkg{xparse}, \pkg{xtemplate},
\pkg{l3keys2e}\footnote{\CTANurl{l3packages}},
\pkg{zref-abspage}\footnote{\CTANurl{oberdiek}} and
\needpackage{translations}~\cite{pkg:translations}.

\section{About}
\begin{cnltxquote}[\cite{wikipedia}]
  Acronyms and initialisms are abbreviations formed from the initial
  components in a phrase or a word.  These components may be individual
  letters (as in CEO) or parts of words (as in Benelux and Ameslan).  There is
  no universal agreement on the precise definition of the various terms nor on
  written usage.
\end{cnltxquote}
After \wikipedia{} told us what acronyms are and we won't confuse them with
units or other kinds of abbreviations -- why would we need another package for
them?  There are several already: \pkg{acronym}~\cite{pkg:acronym},
\pkg{acromake}~\cite{pkg:acromake}, \pkg{acroterm}~\cite{pkg:acroterm}, the
abbreviations package \pkg{abbrevs}~\cite{pkg:abbrevs} (the current version
1.4 has a bug\footnote{see \url{http://tex.stackexchange.com/q/59840/5049} for
  solutions.}, though), the nomenclature package
\pkg{nomencl}~\cite{pkg:nomencl}, and of course the mighty
\pkg{glossaries}~\cite{pkg:glossaries}. So there is really no \emph{need} for
a new package.

On the other hand \pkg{acronym}, the best of the acronym specific packages,
has one or two shortcomings and sometimes using \pkg{glossaries} seems a bit
of an overkill (or simply inconvenient as one has to run
\code{makeglossaries}, \code{makeindex} or \code{xindy}, then\footnote{Rumour
  has it there is going to be a version that can be used without running an
  external program}). So \acro\ stands somewhere in between (but closer to
\pkg{acronym}).

The main reason for the existance of \acro\ is a question on \acs{tex.sx}%
\footnote{\url{http://tex.stackexchange.com/q/59449/5049}} which intrigued me
and in consequence led to \acro\ and it's option \option{single}.

\acro\ has many similarities with the \pkg{acronym} package.  In fact, quite
some macros have the same name and meaning\footnote{\emph{Not} in the sense of
  \cs*{meaning}!}.

Please take a minute to think and decide which package will suit your needs
best.  Are you planning to add a glossary to your book?  You should probably
go with \pkg{glossaries}, then.  Are you planning to add a nomenclature?  You
may want to choose \pkg{nomencl} (or again: \pkg{glossaries}) and so on.
\acro\ does a good job for lists of abbreviations.

\section{News}
\subsection{Version 1.6}
Support for versions~0.* has been dropped.

\section{Basics}
\subsection{Creating New Acronyms}
\noindent\changedversion{1.0}Acronyms are created with the command
\cs{DeclareAcronym} that can only be used in the preamble.
\begin{commands}
  \command{DeclareAcronym}[\marg{id}\marg{list of keys}]
    The basic command for declaring an acronym.
\end{commands}
This command understands a number of keys which are listed below.  Some of
them are not described immediately but at appropriate places in the
documentation.
\begin{options}
  %% short
  \keyval{short}{text}\Default!
    the short form of the acronym.
  %% long
  \keyval{long}{text}\Default!
    the long form of the acronym.
  %% short-plural
  \keyval{short-plural}{text}\Default{s}
    the plural ending appended to the short form.
  %% long-plural
  \keyval{long-plural}{text}\Default{s}
    the plural ending appended to the long form.
  %% long-plural-form
    \keyval{long-plural-form}{text}
      plural long form of the acronym; replaces the long form when used
      instead of appending the plural ending.
  %% list
  \keyval{list}{text}
    \sinceversion{1.4}if specified this will be written in the list as
    description instead of the long form.
  %% short-indefinite
  \keyval{short-indefinite}{text}\Default{a}
    \sinceversion{1.2}indefinite article for the short form.
  %% long-indefinite
  \keyval{long-indefinite}{text}\Default{a}
    \sinceversion{1.2}indefinite article for the long form.
  %% long-pre
  \keyval{long-pre}{text}
    \sinceversion{1.1}\meta{text} is prepended to the long form in the text
    but not in the list of acronyms.
  %% long-post
  \keyval{long-post}{text}
    \sinceversion{1.1}\meta{text} is appended to the long form in the text but
    not in the list of acronyms.
  %% alt
  \keyval{alt}{text}
    alternative short form.
  %% alt-indefinite
  \keyval{alt-indefinite}{text}\Default{a}
    \sinceversion{1.2}indefinite article for the alternative form.
  %% extra
  \keyval{extra}{text}
    extra information to be added in the list of acronyms.
  %% foreign
  \keyval{foreign}{original long form}
    \sinceversion{1.3}can be useful when dealing with acronyms in foreign
    languages, see section~\ref{ssec:foreign} for details.
  %% sort
  \keyval{sort}{text}
    if used the acronym will be sorted according to this key instead of its
    \acs{id}.
  %% class
  \keyval{class}{text}
    the class the acronym belongs to.
  %% cite
  \keylit{cite}{\oarg{prenote}\oarg{postnote}\marg{citation keys}}
    a citation that is printed to the acronym according to an option explained
    later.
  %% short-format
  \keyval{short-format}{\TeX{} code}
    the format used for the short form of the acronym.
  %% long-format
  \keyval{long-format}{\TeX{} code}
    the format used for the long form of the acronym.
  %% first-long-format
  \keyval{first-long-format}{\TeX{} code}
    the format used for the first long form of the acronym as set with \cs{ac},
    \cs{acf} or \cs{acflike} and their uppercase, plural and indefinite forms.
  %% pdfstring
  \keylit{pdfstring}{\Marg{\meta{text}/\meta{plural ending}}}
    used as \acs{pdf} string replacement in bookmarks when used together with the
    \pkg{hyperref} package.  The appended plural ending is optional.  If you
   leave it (\emph{and} the \code{/}) the default ending is used.
  %% accsupp
  \keyval{accsupp}{text}
    sets the \code{ActualText} key as presented by the \pkg{accsupp} package
    for the acronym.
  %% index-sort
  \keyval{index-sort}{text}
    \sinceversion{1.1}If you use the package option \option{index} every
    occurrence of an acronym is recorded to the index and sorted by its
    \ac{id} or (if set) by the value of the \option{sort} key.  This key
    allows to set an individual sorting option for the index.  See
    section~\ref{ssec:index} for details.
  %% index
  \keyval{index}{text}
    \sinceversion{1.1}This key allows to overwrite the automatic index entry
    with an arbitrary one.  See section~\ref{ssec:index} for details.
 %% index-cmd
  \keyval{index-cmd}{text}
    \sinceversion{1.1}This key let's you set an individual index creating
    command for this acronym.  It should be a command that takes one mandatory
    argument.  See section~\ref{ssec:index} for details.
\end{options}

In its simplest form an acronym needs a short and a long form.  Please note
that both keys \emph{must} be set and that the \key{short} key \emph{must}
always be the \emph{first} key that is set.
\begin{sourcecode}
  % preamble:
  \DeclareAcronym{test}{
    short = ST ,
    long  = Some Test
  }
\end{sourcecode}
This creates the acronym ``\acs{test}'' with the \acs{id} ``test'' and the
long form ``\acl{test}.''

The \option{format} key allows you to choose a specific format for the short
form of an acronym:
\begin{sourcecode}
  % preamble:
  \DeclareAcronym{ot}{
    short        = ot ,
    long         = Other Test ,
    short-format = \scshape
  }
\end{sourcecode}
The short form now looks like this: \acs{ot}.

The \option{cite} key needs a bit explaining.  It expects arguments like the
standard \cs*{cite} command, \latin{i.e.}, two optional arguments setting the
\meta{prenote} and \meta{postnote} and one mandatory argument setting the
citation key.
\begin{sourcecode}
  % preamble:
  \DeclareAcronym{ny}{
    short        = NY ,
    short-plural = ,
    long         = New York ,
    long-plural  = ,
    cite         = {NewYork} 
  }
\end{sourcecode}

\begin{sourcecode}[sourcecode-options={style=cnltx-bibtex}]
  % bib file for use with biber/biblatex:
  @online{NewYork,
    author  = {Wikipedia},
    title   = {New York City},
    urldate = {2012-09-27},
    url     = {http://en.wikipedia.org/wiki/New_York_City},
    year    = {2012}
  }
\end{sourcecode}
The first appearance now looks as follows\footnote{The appearance of the
  citation of course depends on the citation style you're using.}: \acf{ny}.

\subsection{Using the Acronyms -- the Commands}
Acronyms are used with with one of the following commands:
\begin{commands}
  \command{ac}[\sarg\marg{id}]
    basic command; the first output is different from subsequent ones.
  \command{Ac}[\sarg\marg{id}]
    same as \cs{ac} but capitalizes the first letter of the long form.
  \command{acs}[\sarg\marg{id}]
    \h{s}hort form; the actual acronym.
  \command{acl}[\sarg\marg{id}]
    \h{l}ong form; the meaning of the acronym.
  \command{Acl}[\sarg\marg{id}] 
    same as \cs{acl} but capitalizes first letter.
  \command{aca}[\sarg\marg{id}]
    \h{a}lternative short form as specified in the \option{alt} key of
    \cs{DeclareAcronym}; if it hasn't been specified this is identical to
    \cs{acs}.
  \command{acf}[\sarg\marg{id}]
    first form; output like the first time \cs{ac} is output.
  \command{Acf}[\sarg\marg{id}]
    same as \cs{acf} but capitalizes first letter of the long form.
  \command{acp}[\sarg\marg{id}]
    \h{p}lural form of \cs{ac};
  \command{Acp}[\sarg\marg{id}]
    same as \cs{acp} but capitalizes first letter of the long form.
  \command{acsp}[\sarg\marg{id}]
    plural form of \cs{acs};
  \command{aclp}[\sarg\marg{id}]
    plural form of \cs{acl};
  \command{Aclp}[\sarg\marg{id}]
    same as \cs{aclp} but capitalizes first letter.
  \command{acap}[\sarg\marg{id}]
    plural form of \cs{aca};
  \command{acfp}[\sarg\marg{id}]
    plural form of \cs{acf};
  \command{Acfp}[\sarg\marg{id}]
    same as \cs{acfp} but capitalizes first letter of the long form.
\end{commands}
If an acronym is used the first time with \cs{ac} its output is different from
subsequent uses.  To be clear on this: the first time!  If the acronym has
been used with \emph{any} of the output commands before it is \emph{not} the
first time any more.

\sinceversion{0.5}If you use the starred variant an acronym will not be marked
as used.  This proves useful if an acronym is typeset in a section title, for
example, since then the appearance in the table of contents won't mark it as
used.

\begin{example}[side-by-side]
  % preamble:
  % \DeclareAcronym{cd}{
  %   short        = cd ,
  %   long         = Compact Disc ,
  %   short-format = \scshape
  % }
  first time: \ac{cd} \\
  second time: \ac{cd} \\
  short: \acs{cd} \\
  alternative: \aca{cd} \\
  first again: \acf{cd} \\
  long: \acl{cd} \\
  short plural: \acsp{cd} \\
  long plural: \aclp{cd}
\end{example}

\subsection{Plural Forms}
If an acronym is defined in the standard way \acro\ uses an `s' that's appended
to both the short and the long form when one of the plural commands is used.
However, that is not always the best solution.  For one thing not all acronyms
may have a plural form.  Second, the plural form especially of the long forms
may be formed differently.  And third, other languages can have other plural
endings.

For these reasons \cs{DeclareAcronym} can get the following keys:
\begin{options}
  \keyval{short-plural}{text}\Default{s}
    The plural ending of the short form.
  \keyval{long-plural}{text}\Default{s}
    The plural ending of the long form.
  \keyval{long-plural-form}{text}
    An alternative plural form for the long form.
\end{options}
These keys are optional.  If they're not used, the default setting is
\code{s}.  If you use \option{long-plural-form} the long form will be replaced
by the specified plural form when necessary.

Suppose we define the following acronyms:
\begin{sourcecode}
  \DeclareAcronym{cd}{
    short        = cd ,
    long         = Compact Disc ,
    short-format = \scshape
  }
  \DeclareAcronym{ny}{
    short        = NY ,
    short-plural = ,
    long         = New York ,
    long-plural  =
  }
  \DeclareAcronym{sw}{
    short       = SW ,
    long        = Sammelwerk ,
    long-plural = e
  }
  \DeclareAcronym{MP}{
    short            = MP ,
    long             = Member of Parliament ,
    long-plural-form = Members of Parliament
  }
\end{sourcecode}
These acronyms now have the following plural appearances:
\begin{example}[side-by-side]
  \acsp{cd}, \aclp{cd} \\
  \acsp{ny}, \aclp{ny} \\
  \acsp{sw}, \aclp{sw} \\
  \acsp{MP}, \aclp{MP}
\end{example}

\subsection{Alternative Short Forms}
For some acronyms it might be useful to have alternative forms.  For this
\cs{DeclareAcronym} has another key:
\begin{options}
 \keyval{alt}{text}
   Alternative short form.
\end{options}
\begin{example}
  % preamble:
  % \DeclareAcronym{jpg}{
  %   short = JPEG ,
  %   alt   = JPG ,
  %   long  = Joint Photographic Experts Group
  % }
  default: \acs{jpg} \\
  alt.: \aca{jpg}
\end{example}
The alternative form uses the same plural ending as the default short form and
is formatted in the same way.

\subsection{Extra Information for the List Entry}
Of course you can print a list of acronyms where their meaning is explained.
Sometimes it can be useful to add additional information there.  This is done
with another key to \cs{DeclareAcronym}:
\begin{options}
  \keyval{extra}{text}
    Additional information for the list of acronyms.
\end{options}
These information will only be displayed in the list.  See
section~\ref{sec:print_lists} for the impact of the following example.

\begin{example}
  % preamble:
  % \DeclareAcronym{nato}{
  %   short        = nato ,
  %   long         = North Atlantic Treaty Organization ,
  %   extra        = \textit{deutsch}: Organisation des Nordatlantikvertrags ,
  %   short-format = \scshape
  % }
  The \ac{nato} is an intergovernmental military alliance based on the
  North Atlantic Treaty which was signed on 4~April 1949. \ac{nato}
  headquarters are in Brussels, Belgium, one of the 28 member states
  across North America and Europe, the newest of which, Albania and
  Croatia, joined in April 2009.
\end{example}

\subsection{Foreign Language Acronyms}\label{ssec:foreign}
\noindent\sinceversion{1.3}I repeatedly read the wish for being able to add
translations to acronyms when the acronyms stem from another language than the
document language, \latin{i.e.}, something like the following in a German 
document:
\begin{example}[side-by-side]
  \ac{ecu}\\
  \ac{ecu}
\end{example}
That's why I decided to add the following key:
\begin{options}
  \keyval{foreign}{original long form}
    A description for an acronym originating in another language than the
    document language.
\end{options}

Here is the definition of the above mentioned \ac{ecu} acronym:
\begin{sourcecode}
  \DeclareAcronym{ecu}{
    short   = ECU ,
    long    = Steuerger\"at ,
    foreign = Electronic Control Unit
  }
\end{sourcecode}
As you have seen this adds the \option{foreign} entry to the first appearance
of an acronym.  It is also added in parentheses to the list of acronyms after
the \option{long} entry.  Actually the entry there is the argument to the
following command:
\begin{commands}
  \command{acroenparen}[\marg{argument}]
    Places \meta{argument} in parentheses: \cs{acroenparen}\Marg{example}:
    \acroenparen{example}.  See page~\pageref{key:list-foreign-format} for a
    way to customize this other than redefining it.
\end{commands}

\section{Additional Commands and Possibilities}
\subsection{Indefinite Forms}
\noindent\sinceversion{1.2}%
Unlike many other languages\footnote{Let's better say: unlike the other
  languages where I know at least the basics.} in English the indefinite
article is not determined by the grammatical case, gender or number but by the
pronounciation of the following word.  This means that the short and the long
form of an acronym can have different indefinite articles.  For these cases
\acro\ offers the keys \option{short-indefinite}, \option{alt-indefinite} and
\option{long-indefinite} whose default is \code{a}.  For every lowercase
singular command two alternatives exist, preceded by \code{i} and \code{I},
respectively, which output the lowercase and uppercase version of the
corresponding indefinite article.

\begin{example}
  % preamble:
  % \DeclareAcronym{ufo}{
  %   short           = UFO ,
  %   long            = unidentified flying object ,
  %   long-indefinite = an
  % }
  \Iac{ufo}; \iacs{ufo}; \iacl{ufo}
\end{example}

\subsection{Uppercasing}
\begin{commands}
  \command{acfirstupper}[\marg{token list}]
     \sinceversion{1.3e}This command uppercases the first token in \meta{token
       list}.  The command is less powerful than \cs{makefirstuc} that is
     provided by the \pkg{mfirstuc} package~\cite{pkg:mfirstuc} but it is
     expandable.  Obvious downsides are for example that it does not uppercase
     accented letters.
\end{commands}

\subsection{Simulating the First Appearance}
\noindent\sinceversion{1.2}%
Users told me\footnote{Well -- one, to be precise ;)} that there are cases
when it might be useful to have the the acronym typeset according to the
\option{first-style} but with another text than the long form.  For such cases
\acro\ offers the following commands.
\begin{commands}
  \command{acflike}[\sarg\marg{id}\marg{instead of long form}]
    Write some alternative long form for acronym \meta{id} as if it were the
    first time the acronym was used.
  \command{acfplike}[\sarg\marg{id}\marg{instead of long form}]
    Plural form of \cs{acflike}.
\end{commands}

\begin{example}[side-by-side]
  \acsetup{first-style=footnote}
  \acflike{ny}{the big apple}
\end{example}

The plural ending in \cs{acfplike} is only appended to the short form.  It
makes no sense to append it to the text that is inserted manually anyway.
Note that whatever text you're inserting might be gobbled depending on the
\option{first-style} you're using.

\subsection{Using Classes}
The acronyms of \acro\ can be divided into different classes.  This doesn't
change the output but allows different acronym lists, see
section~\ref{sec:print_lists}.  For this \cs{DeclareAcronym} has an additional
key:
\begin{options}
  \keyval{class}{text}
    Associated class for an acronym.
\end{options}

This might be useful if you can and want to divide your acronyms into
different types, technical and grammatical ones, say, that shall be listed in
different lists.

\begin{example}[side-by-side]
  % preamble:
  % \DeclareAcronym{la}{
  %   short        = LA ,
  %   short-plural = ,
  %   long         = Los Angeles ,
  %   long-plural  = ,
  %   class        = city
  % }
  % \DeclareAcronym{ny}{
  %   short        = NY ,
  %   short-plural = ,
  %   long         = New York ,
  %   long-plural  = ,
  %   class        = city ,
  %   cite         = NewYork
  % }
  \acl{la} (\acs{la}) \\
  \acl{ny} (\acs{ny})
\end{example}

\subsection{Reset or Mark as Used, Test if Acronym Has Been Used}

If you want for some reason to fool \acro\ into thinking that an acronym is
used for the first time you can call one of these commands:
\begin{commands}
  \command{acreset}[\marg{comma separated list of ids}]
    \sinceversion{0.5}This will reset a used acronym such that the next use of
    \cs{ac} will again print it as if it were used the first time.  This will
    \emph{not} remove an acronym from being printed in the list if it actually
    \emph{has} been used before.
  \command{acresetall}
    Reset all acronyms.
  \command{acifused}[\marg{id}\marg{true}\marg{false}]
    \sinceversion{1.3e}This command tests if the acronym with \ac{id}
    \meta{id} has already been used and either puts \code{true} or
    \code{false} in the input stream.
\end{commands}
\begin{example}[side-by-side]
  \acreset{ny}\ac{ny}
\end{example}
Beware that both commands act \emph{globally}!  There are also commands that
effectively do the opposite of \cs{acreset}, \latin{i.e.}, mark acronyms as
used:
\begin{commands}
  \command{acuse}[\marg{comma separated list of ids}]
    \sinceversion{0.5}This has the same effect as if an acronym had been used
    twice, that is, further uses of \cs{ac} will print the short form and the 
    acronym will in any case be printed in the list (as long as its class is
    not excluded).
  \command{acuseall}
    \sinceversion{0.6a}Mark all acronyms as used.
\end{commands}

\subsection{\cs*{ac} and Friends in \acs*{pdf} Bookmarks}
\noindent\sinceversion{0.5}\acro's commands usually are not expandable which
means they'd leave unallowed tokens in \acs{pdf} bookmarks.  \pkg{hyperref}
offers \cs*{texorpdfstring} to circumvent that issue manually but that isn't
really a nice solution.  What's the point of having macros to get output for
you if you have to specify it manually after all?

That is why \acro\ offers a preliminary solution for this.  In a bookmark
every \cs{ac} like command falls back to a simple text string typesetting what
\cs{acs} would do (or \cs{acsp} for plural forms).  These text strings both
can accessed manually and can be modified to an output reserved for \acs{pdf}
bookmarks.

\begin{commands}
  \command{acpdfstring}[\marg{id}]
    Access the text string used in \acs{pdf} bookmarks.
  \command{acpdfstringplural}[\marg{id}]
    Access the plural form of the text string used in \acs{pdf} bookmarks.
\end{commands}
\begin{options}
  \keylit{pdfstring}{\Marg{\meta{pdfstring}/\meta{plural ending}}}
    Key for \cs{DeclareAcronym} to declare a custom text string for \acs{pdf}
    bookmarks.  The plural ending can be set optionally.
  \keyval{accsupp}{text}
    \sinceversion{1.0}Key for \cs{DeclareAcronym} to set the \code{ActualText}
    property of \cs*{BeginAccSupp} (see \pkg{accsupp}'s documentation for
    details) to be used for an acronym.  It only has an effect when the
    package option \option{accsupp} is used, too.
\end{options}
  
For example the \acs{pdf} acronym used in the title for this section is defined
as follows:
\begin{sourcecode}
  \DeclareAcronym{pdf}{
    short     = pdf ,
    long      = Portable Document Format ,
    format    = \scshape ,
    pdfstring = PDF ,
    accsupp   = PDF
  }
\end{sourcecode}
 
This also demonstrates the \option{accsupp} key.  For this to work you need to
use the \emph{package option} \option{accsupp}, too, which will load the
package \pkg{accsupp}.  Then the key \option{accsupp} will set the
\code{ActualText} property of \cs*{BeginAccSupp}.  Please refer to
\pkg{accsupp}'s documentation for details.  To see its effect copy \ac{pdf}
and paste it into a text file.  You should get uppercase letters instead of
lowercase ones.

\subsection{Adding Acronyms to the Index}\label{ssec:index}
\noindent\sinceversion{1.1}\acro\ has the package option \option{index}.  If it is
used an index entry will be recorded every time an \emph{unstarred} acronym
command is used.  The index entry will be \code{\meta{id}@\meta{short}},
\code{\meta{sort}@\meta{short}} if the \option{sort} key has been set,
\code{\meta{index-sort}@\meta{short}} if the \option{index-sort} has been set,
or \meta{index} if the key \option{index} has been set for the specific
acronym.  The short versions appearing there are formatted according to the
chosen format of the corresponding acronym, of course.

This document demonstrates the feature.  You can find every acronym that has
been declared in the index.  In order to allow flexibility the indexing
command can be chosen both globally via package option and individually for
every acronym.  This would allow to add acronyms to a specific index if more
than one index is used, for example with help of the \pkg*{imakeidx} package.

I'm not yet convinced this is a feature many people if anyone needs and if
they do if it is flexible enough.  If you have any thoughts on this I'd
appreciate an email.

\section{Printing the List}\label{sec:print_lists}
\noindent\changedversion{1.0}Printing the whole list of acronyms is easy: just
place \cs{printacronyms} where ever you want the list to be.
\begin{commands}
  \command{printacronyms}[\oarg{options}]
    Print the list of acronyms.
\end{commands}
The commands takes a few options, namely the following ones:
\begin{options}
  \keyval{include-classes}{list of classes}
    Takes a comma-separated list of the classes of acronyms that should be in
    the list.
  \keyval{exclude-classes}{list of classes}
    Takes a comma-separated list of the classes of acronyms that should
    \emph{not} be in the list.
  \keyval{name}{name of the list}
    sets the name for the list.
  \keyval{heading}{sectioning command without leading backslash}%
    \Default{section*}
    \changedversion{1.3}Sets the sectioning command for the heading of the
    list.  A special value is \code{none} which suppresses the heading.
  \keybool{sort}\Default{true}
    \sinceversion{1.3}Set sorting for this list only.
\end{options}
\begin{sourcecode}
  \acsetup{extra-style=comma}
  \printacronyms[exclude-classes=city]
 
  \printacronyms[include-classes=city,name={City Acronyms}]
\end{sourcecode}
\acsetup{extra-style=comma}
\printacronyms[exclude-classes=city]

\printacronyms[include-classes=city,name={City Acronyms}]

You can see that the default layout is a \code{description} list with a
\cs*{section}\sarg\ title.  Both can be changed, see
section~\ref{sec:customization}.

The command \cs{printacronyms} needs two \LaTeX{} runs.  This is a precaution
to avoid error messages with a possibly empty list.  But since almost all
documents need at least two runs and often are compiled much more often than
that, this fact shouldn't cause too much inconvenience.

\section{Options and Customization}\label{sec:customization}
\subsection{General Options}
There are a few options which change the general behaviour of \acro.
\default{Underlined} values are used if no value is given.
\begin{options}
  %%
  % \keychoice{version}{0,1}\Default{1}
  %   Provide backwards compatibility for documents set with \acro\ in a version
  %   prior to v1.0.
  %%
  \keychoice{messages}{silent,loud}\Default{loud}
    \sinceversion{1.6}Setting \keyis{messages}{silent} will turn all of
    \acro's error messages into warnings and all of \acro's warnings into info
    messages.  Be sure to check the log file carefully if you decide to set
    this option.
  \keybool{single}\Default{false}
    If set to \code{true} an acronym that's used only once (with \cs{ac}) in a
    document will only print the long form and will not be printed in the list.
  %%
  \keybool{hyperref}\Default{false}
    If set to \code{true} the short forms of the acronyms will be linked to
    their list entry.
  %%
  \keybool{label}\Default{false}
    \sinceversion{1.5}If set to \code{true} this option will place
    \cs*{label}\Marg{\meta{prefix}\meta{id}} the first time the acronym with
    \ac{id} \meta{id} is used.
  %%
  \keyval{label-prefix}{text}\Default{ac:}
    \sinceversion{1.5}The prefix for the \cs*{label} that is placed when
    option \keyis{label}{true} is used.
  %%
  \keybool{record-pages}\Default{true}
    Since \acro\ can handle arabic, roman and Roman page numbers but
    \emph{not} any other kind of numbering this option allows to turn the page
    number recording off for these cases as it would lead to errors else.
    This affects the whole document and can only be set in the preamble!  It
    means you cannot have page numbers in the list of acronyms in this case.
    Or rather: you can if you use \keyis{pages}{first}.
  %%
  \keybool{only-used}\Default{true}
    This option is \code{true} as default.  It means that only acronyms that
    are actually used in the document are printed in the list.  If
    \code{false}, all acronyms defined with \cs{DeclareAcronym} will be
    written to the list.
  %%
  \keychoice{mark-as-used}{first,any}\Default{any}
    \sinceversion{1.2}%
    This option determines wether an acronym is mark as used when the
    \emph{first} form is used the first time (with \cs{ac}, \cs{acf} or
    \cs{acflike} and their uppercase, plural and indefinite forms) or when any
    of the \cs{ac}-like commands is used.   
  %%
  \keybool{macros}\Default{false}
    If set to \code{true} this option will create a macro \cs*{\meta{id}} for
    each acronym as a shortcut for \cs{ac}\marg{id}.  Already existing macros
    will \emph{not} be overwritten.
  %%
  \keybool{xspace}\Default{false}
    \sinceversion{0.6}If set to \code{true} this option will append
    \cs*{xspace} from the \pkg*{xspace} package to the commands created with
    the \option{macros} option.
  %%
  \keybool{strict}\Default{false}
    If set to \code{true} and the option \keyis{macros}{true} is in effect
    then already existing macros will be overwritten.
  %%
  \keybool{sort}\Default{true}
    If set to \code{true} the acronym list will be sorted automatically.  The
    entries are sorted by their \acs{id} ignoring upper and lower case.  This
    option needs the experimental package \pkg{l3sort} (from the
    \pkg{l3experimental} bundle) and can only be set in the preamble.
  %%
  \keychoice{cite}{\default{all},first,none}\Default{first}
    This option decides whether citations that are added via \option{cite} are
    added to each first, every or no appearance of an acronym.
  %%
  \keyval{cite-cmd}{control sequence}\Default{\cs*{cite}}
    This option determines which command is used for the citation.  Each
    citation command that takes the cite key as argument is valid, for example
    \pkg*{biblatex}'s \cs*{footcite}.
  %%
  \keyval{cite-space}{code}\Default{\cs*{nobreakspace}}
    Depending on the citation command in use a space should be inserted before
    the citation or maybe not (e.g.\ \cs*{footcite}\ldots).  This option
    allows you to set this.  Actually it can be used to place arbitrary code
    right before the citation.
  %%
  \keybool{index}\Default{false}
    \sinceversion{1.1}If set to \code{true} an index entry will be recorded
    every time an \emph{unstarred} acronym command is used for the
    corresponding acronym.
  %%
  \keyval{index-cmd}{control sequence}\Default{\cs*{index}}
    \sinceversion{1.1}Chooses the index command that is used when option
    \option{index} has been set to \code{true}.
  %%
  \keybool{accsupp}\Default{false}
    \sinceversion{1.0}Activates the access support as provided by the
    \pkg{accsupp} package.
  %%
  \keyval{uc-cmd}{control sequence}\Default{\cs{acfirstupper}}
    The command that is used to capitalize the first word in the \cs{Ac} and
    the like commands.  You can change it to another one like for example
    \cs*{makefirstuc}\footnote{from the \pkg{mfirstuc} package} or
    \cs*{MakeTextUppercase}\footnote{from the \pkg*{textcase} package}.
\end{options}
 
All options of this and the following sections can be set up either as package
options or via the setup command:
\begin{commands}
  \command{acsetup}[\marg{options}]
   Set up \acro\ anywhere in the document.  Or separate package loading from
   setup.
\end{commands}

\begin{example}
  % with \acsetup{macros}
  we could have used these before: \nato, \ny
\end{example}

\subsection{Options Regarding Acronyms}
The options described in this section all influence the layout of one of the
possible output forms of the acronyms.
\begin{options}
  %%
  \keyval{short-format}{format}\Default
    Sets a format for all short forms. For example
    \keyis{short-format}{\cs*{scshape}} would print all short forms in small
    caps.
  %%
  \keyval{long-format}{format}\Default
    The same for the long forms.
  %%
  \keyval{foreign-format}{format}\Default
    \sinceversion{1.3}%
    The format for the \option{foreign} entry when it appears as part of the
    first appearance of an acronym.
  %%
  \keyval{first-long-format}{format}\Default
    \sinceversion{1.2}%
    The format for the long form on first usage (with \cs{ac}, \cs{acf} or
    \cs{acflike} and their uppercase, plural and indefinite forms).
  %%
  \keyval{list-short-format}{format}\Default
    \sinceversion{1.1}An extra format for the short entries in the list.  If
    not used this is the same as \option{short-format}.  Please be aware that
    a call of \option{short-format} after this one will overwrite it again.
  %%
  \keyval{list-long-format}{format}\Default
    An extra format for the long entries in the list.  If not used this is the
    same as \option{long-format}.  Please be aware that a call of
    \option{long-format} after this one will overwrite it again.
  %%
  \keyval{list-foreign-format}{format}\Default{\cs{acroenparen}}
    \label{key:list-foreign-format}The format for the \option{foreign} entry
    as it appears in the list.  This may be code that ends with a macro that
    takes a mandatory argument.
  %%
  \keyval{extra-format}{format}\Default
    The same for the extra information.
  %%
  \keychoice{first-style}{default,plain,empty,square,short,reversed,plain-reversed,footnote,sidenote}\Default{default}
    \changedversion{1.1}The basic style of the first appearance of an
    acronym.  The value \code{sidenote} needs the command \cs*{sidenote} to be
    defined for example by the \pkg*{sidenotes} package.
  %%
  \keychoice{extra-style}{default,plain,comma,paren,bracket}\Default{default}
    Defines the way the extra information is printed in the list.
  %%
  \keyval{plural-ending}{tokenlist}\Default{s}
    With this option the default plural ending can be set.
\end{options}
 
\begin{example}[side-by-side]
  % (Keep in mind that we're in
  % a minipage here!)
  \acsetup{first-style=empty}
  \acf{ny} \\
  \acsetup{first-style=footnote}
  \acf{ny} \\
  \acsetup{first-style=square}
  \acf{ny} \\
  \acsetup{first-style=short}
  \acf{ny} \\
  \acsetup{first-style=reversed}
  \acf{ny} \\
  \acsetup{first-style=plain}
  \acf{ny} \\
  \acsetup{first-style=plain-reversed}
  \acf{ny}
\end{example}

\subsection{Options Regarding the List}
\begin{options}
  %%
  \keychoice{page-ref}{none,plain,comma,paren}\Default{none}
    If this option is set to a value other than \code{none} the page numbers
    of the an acronym appeared on are printed in the list.  Please note that
    this is an experimental feature and might fail in quite a number of cases.
    If you notice anything please send me an email!
  %%
  \keychoice{pages}{all,first}\Default{all}
    \sinceversion{1.5}If the option \option{page-ref} has any value other than
    \code{none} this option determines wether all usages of the acronyms are
    listed or only the first time.  Implicitly sets \keyis{label}{true}.
  \keyval{page-name}{page name}\Default{p.\cs*{@}\cs*{,}}
    The ``name'' of the page label.  This is automatically translated to the
    active language. However for the time being there are many translations
    missing, yet.  Please notify me if you find your language missing.
  %%
  \keyval{pages-name}{page name plural}\Default{pp.\cs*{@}\cs*{,}}
    \sinceversion{1.0}The ``name'' of the page label when there are more than
    one page.  This is automatically translated to the active language.
    However for the time being there are many translations missing, yet.
    Please notify me if you find your language missing.
  %%
  \keybool{following-page}\Default{false}
    \sinceversion{1.3}If set to \code{true} a page range in the list of
    acronyms that consists of two pages will be written by the first page and
    an appended \code{f}. This depends on the option \option{next-page}.
  %%
  \keybool{following-pages}\Default{false}
    \sinceversion{1.3}If set to \code{true} a page range in the list of
    acronyms that set consists of more than two pages will be written by the
    first page and an appended \code{ff}. This depends on the option
    \option{next-pages}.
  %%
  \keyval{next-page}{text}\Default{\cs*{,}f.\cs*{@}}
    \sinceversion{1.0}Appended to a page number when \option{following-page}
    is set to \code{true} and the range is only 2 pages long.  This is
    automatically translated to the active language.  However, for the time
    being there are many translations missing, yet.  Please notify me if you
    find your language missing.
  %%
  \keyval{next-pages}{text}\Default{\cs*{,}ff.\cs*{@}}
    \sinceversion{1.0}Appended to a page number when \option{following-pages}
    is set to \code{true} and the range is more than 2 pages long.  This is
    automatically translated to the active language.  However, for the time
    being there are many translations missing, yet.  Please notify me if you
    find your language missing.
  %%
  \keychoice{list-type}{table|\meta{list}}\Default{description}
    This option let's you choose how the list is printed.  \meta{list} can be
    any valid list like \env{itemize} or \env{description}.
  %%
  \keychoice{list-style}{list,tabular,longtable,extra-tabular,extra-longtable,%
    extra-tabular-rev,extra-longtable-rev}\Default{list}
    If you choose \keyis{list-type}{table} you have to specify which kind of
    table should be used.  If you choose \code{longtable},
    \code{extra-longtable} or \code{extra-longtable-rev} you have to load
    \pkg{longtable} in your preamble.  The values \code{extra-longtable} and
    \code{extra-longtable-rev} put the extra information in a column of its
    own.
  %%
  \keychoice{list-heading}{chapter,chapter*,section,section*,subsection,%
    subsection*,addchap,addsec,none}\null
  \Default{section*}
    \changedversion{1.3}The heading type of the list. The last two only work
    with a \KOMAScript{} class that also defines the appropriate command.  A
    special value is \code{none} which suppresses the heading.
  %%
  \keyval{list-name}{list name}\Default{Acronyms}
    The name of the list.  This is what's written in the list-heading.  This
    is automatically translated to the active language.  However, for the time
    being there are many translations missing, yet.  Please notify me if you
    find your language missing.
  %%
  \keyval{list-table-width}{dimension}\Default{.7\cs*{linewidth}}
    This has only an effect if you chose \key{list-type}{table}.  The
    \emph{second} column (or the third if you choose one of the \code{-rev}
    styles) of the table is a \code{p} column whose width can be specified
    with this option.
  %%
  \keybool{list-caps}\Default{false}
    Print the first letters of the long form capitalized.
 \end{options}
  
If you for example have loaded \pkg{enumitem} you can define a custom list for
the acronym list:
\begin{sourcecode}
  % preamble:
  % \usepackage{enumitem}
  \newlist{acronyms}{description}{1}
  \newcommand*\addcolon[1]{#1:}
  \setlist[acronyms]{
    labelwidth=3em,
    leftmargin=3.5em,
    noitemsep,
    itemindent=0pt,
    font=\addcolon}
  \acsetup{list-type=acronyms,hyperref=false,extra-style=comma}
  \printacronyms
\end{sourcecode}
\newlist{acronyms}{description}{1}
\newcommand*\addcolon[1]{#1:}
\setlist[acronyms]{
  labelwidth=3em,
  leftmargin=3.5em,
  noitemsep,
  itemindent=0pt,
  font=\addcolon}
\acsetup{list-type=acronyms,extra-style=comma}
\printacronyms

\section{About Page Ranges}
If you enable the \option{page-ref} option \acro\ adds page numbers to the list
of acronyms.  In version~0.\versionstar{} it would add a page reference for an
acronym in the list of acronyms that used \cs*{pageref} to refer to the first
appearance of an acronym.  This is retained using \keyis{version}{0}.
Version~1.0 uses a different approach that doesn't use a label but instead
will list \emph{all} pages an acronym appeared on.  With \pkg{hyperref} the
pages are referenced using \cs*{hyperpage}.

There are some options that control how this list will be typeset, e.g.,
\option{following-page}, \option{next-pages} or the option \option{page-ref}
itself.  It is important to mention that the page list will always take at
least two compilation runs until changes in the options or the actual page
numbers affect it.  This is due to the fact that the updated sequence is first
written to the \code{aux} file and only read in during the next run.


\section{Language Support}
\acro\ detects if packages \pkg*{babel} or \pkg*{polyglossia} are being loaded
and tries to adapt certain strings to match the chosen language.  However, due
to my limited language knowledge only a few translations are provided.  I'll
show how the English translations are defined so you can add the translations
to your preamble if needed.  Even better would be you'd send me a short email
at \href{mailto:contact@mychemistry.eu}{contact@mychemistry.eu} with the
appropriate translations for your language and I'll add them to \acro.

\begin{sourcecode}
  \DeclareTranslation{English}{acronym-list-name}{Acronyms}
  \DeclareTranslation{English}{acronym-page-name}{p.}
  \DeclareTranslation{English}{acronym-pages-name}{pp.}
  \DeclareTranslation{English}{acronym-next-page}{f.}
  \DeclareTranslation{English}{acronym-next-pages}{ff.}
\end{sourcecode}

\section{hyperref Support}
The option \keyis{hyperref}{true} adds internal links from all short (or
alternative) forms to their respective list entries.  Of course this only
works if you have loaded the \pkg{hyperref} package in your preamble.  You
should use this option with care: if you don't use \cs{printacronyms} anywhere
this option will result in loads of \pkg{hyperref} warnings.  Also printing
several lists can result in warnings if don't clearly separate the lists into
different classes.  If an acronym appears in more than one list there will
also be more than one hypertarget for this acronym.

Using \pkg{hyperref} will also add \cs*{hyperpage} to the page numbers in the
list (provided they are displayed in the style chosen).  Like with an index
the references will thus not point to the acronyms directly but to the page
they're on.

\appendix

% \section{Commands provided in version~0.\versionstar}
% Prior to version~1.0 the following commands were available.  They are still
% provided if you set the option \keyis{version}{0}.  They will be mentioned
% here shortly but they won't be explained any more.  If you'd like a more
% thorough description ask me for the documentation to version~0.6a.
% \begin{commands}
%   \command{DeclareAcronym}[\sarg\marg{id}\Marg{\meta{short},\meta{pl}}%
%     \oarg{alt.}\Marg{\meta{long},\meta{pl}}\marg{extra}\oarg{class}]
%     \verbcode+% can be used only in preamble+
%   \command{DeclareAcronymFormat}[\marg{id}\marg{format}]
%     \verbcode+% can be used only in preamble+
%   \command{DeclareAcronymCitation}[\marg{id}\oarg{pre}\oarg{post}\marg{cite keys}]
%     \verbcode+% can be used only in preamble+
%   \command{DeclareAcronymPDFString}[\marg{id}\Marg{\meta{pdf entry},\meta{plural ending}}]
%     \verbcode+% can be used only in preamble+
% \end{commands}

\section{All Acronyms Used in this Documentation}\label{sec:documentation_acronyms}
\begin{sourcecode}
  \acsetup{
    list-type    = table,
    list-style   = longtable,
    list-heading = subsection*,
    extra-style  = comma,
    page-ref     = comma
  }
  \printacronyms[name=All Acronyms]
\end{sourcecode}
\acsetup{
  list-type    = table,
  list-style   = longtable,
  list-heading = subsection*,
  extra-style  = comma,
  page-ref     = comma
}
\printacronyms[name=All Acronyms]

\begin{sourcecode}
  \acsetup{
    list-type    = table,
    list-style   = longtable,
    list-heading = subsection*,
    page-ref     = comma
  }
  \printacronyms[include-classes=city,name=City Acronyms]
\end{sourcecode}
\printacronyms[include-classes=city,name=City Acronyms]

\end{document}

