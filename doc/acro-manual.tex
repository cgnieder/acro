% !arara: pdflatex: { interaction: nonstopmode }
% !arara: biber
% arara: pdflatex: { interaction: nonstopmode }
% arara: pdflatex: { interaction: nonstopmode }
% arara: pdflatex: { interaction: nonstopmode }
% --------------------------------------------------------------------------
% the ACRO package
% 
%   Typeset Acronyms
% 
% --------------------------------------------------------------------------
% Clemens Niederberger
% Web:    https://github.com/cgnieder/acro/
% E-Mail: contact@mychemistry.eu
% --------------------------------------------------------------------------
% Copyright 2011--2020 Clemens Niederberger
% 
% This work may be distributed and/or modified under the
% conditions of the LaTeX Project Public License, either version 1.3
% of this license or (at your option) any later version.
% The latest version of this license is in
%   http://www.latex-project.org/lppl.txt
% and version 1.3 or later is part of all distributions of LaTeX
% version 2005/12/01 or later.
% 
% This work has the LPPL maintenance status `maintained'.
% 
% The Current Maintainer of this work is Clemens Niederberger.
% --------------------------------------------------------------------------
% The acro package consists of the files
% - acro.sty, acro.definitions.tex, acro.cfg
% - acro-manual.tex, acro-manual.pdf, acro-manual.cls
% - acro.history, README
% --------------------------------------------------------------------------
% If you have any ideas, questions, suggestions or bugs to report, please
% feel free to contact me.
% --------------------------------------------------------------------------
\PassOptionsToPackage{ngerman,english}{babel}
\PassOptionsToPackage{version=3,deprecation}{acro}
\documentclass{acro-manual}

% \usepackage[]{babel}

\addbibresource{\jobname.bib}
\addbibresource{cnltx.bib}
\begin{filecontents}{\jobname.bib}
@online{wikipedia,
  author   = {Wikipedia},
  title    = {Acronym and initialism},
  urldate  = {2012-06-21},
  url      = {http://en.wikipedia.org/wiki/Acronyms},
  year     = {2012}
}
@online{NewYork,
  author   = {Wikipedia},
  title    = {New York City},
  urldate  = {2012-09-27},
  url      = {http://en.wikipedia.org/wiki/New_York_City},
  year     = {2012}
}
@manual{interface3,
  author    = {{The \LaTeX3 Project Team}} ,
  shorthand = {L3P} ,
  sortname  = {LaTeX3 Project Team} ,
  title     = {The \LaTeX3 Interfaces} ,
  date      = {2015-09-06} ,
  url       = {http://mirrors.ctan.org/macros/latex/contrib/l3kernel/interface3.pdf}
}
\end{filecontents}

% declare acronyms
\DeclareAcronym{cd}{
  short = CD ,
  long  = compact disc
}
\DeclareAcronym{ctan}{
  short     = ctan ,
  long      = Comprehensive \TeX\ Archive Network ,
  format    = \scshape ,
  pdfstring = CTAN ,
  short-acc = CTAN ,
  first-style = short-long ,
  single-style = short
}
\DeclareAcronym{ecu}{
  short   = ECU ,
  long    = Electronic Control Unit ,
  foreign = Steuergerät ,
  foreign-locale = German ,
  foreign-babel = ngerman
}
\DeclareAcronym{id}{
  short        = id ,
  long         = identification string ,
  short-format = \scshape
}
\DeclareAcronym{jpg}{
  short = JPEG ,
  sort  = jpeg ,
  alt   = JPG ,
  long  = Joint Photographic Experts Group
}
\DeclareAcronym{la}{
  short        = LA ,
  short-plural = ,
  long         = Los Angeles,
  long-plural  = ,
  class        = city
}
\DeclareAcronym{lppl}{
  short     = lppl ,
  long      = \unexpanded{\LaTeX} Project Public License ,
  format    = \scshape ,
  pdfstring = LPPL ,
  short-acc = LPPL
}
\DeclareAcronym{MP}{
  short = MP ,
  long  = Member of Parliament ,
  long-plural-form = Members of Parliament
}
\DeclareAcronym{nato}{
  short        = nato ,
  long         = North Atlantic Treaty Organization ,
  foreign      = Organisation des Nordatlantikvertrags ,
  foreign-locale = German ,
  foreign-babel  = ngerman ,
  short-format = \scshape
}
\DeclareAcronym{ny}{
  short        = NY ,
  short-plural = ,
  long         = New York ,
  long-plural  = ,
  class        = city ,
  cite         = NewYork
}
\DeclareAcronym{pdf}{
  short     = pdf ,
  long      = Portable Document Format ,
  format    = \scshape ,
  pdfstring = PDF ,
  short-acc = PDF
}
\DeclareAcronym{sw}{
  short       = SW ,
  long        = Sammelwerk ,
  long-plural = e ,
  class       = exclude
}
\DeclareAcronym{tex.sx}{
  short = \TeX.sx ,
  sort  = TeX.sx ,
  long  = \TeX{} StackExchange
}
\DeclareAcronym{ufo}{
  short           = UFO ,
  long            = unidentified flying object ,
  long-indefinite = an
}

% because the mannual does not activate the `macros' option:
\newcommand*\nato{\ac{nato}}
\newcommand*\ny{\ac{ny}}

\newcommand*\latin[1]{\textit{#1}}

\begin{document}

\begin{bewareofthedog}
  Hi and thanks that you are testing v3.0 of \acro\ before it is released to
  \ac{ctan}. If you want to test the new version use
  \cs*{usepackage}\Oarg{version=3}\Marg{acro}. With \code{version=2} or no
  option at all you get the old version of acro.  Using
  \cs*{usepackage}\Oarg{version=3,deprecation}\Marg{acro} is supposed to give
  as much meaningful warnings and errors as possible.
\end{bewareofthedog}

\part{Get started with \acro}

\section{Licence and requirements}
\license

\section{\acro\ for the impatient}

Acronyms are defined in the preamble via the command
\begin{commands}
  \command{DeclareAcronym}[\marg{id}\marg{properties}]
    where \meta{id} is a unique string to identify the acronym and
    \meta{properties} is a key\slash value list of properties.
\end{commands}
In the document acronyms are used with these commands:
\begin{commands}
  \command{ac}[\marg{id}]
    Prints the acronym \meta{id}, the first time with full description and
    every subsequent use only the abbreviated form.
  \command{Ac}[\marg{id}]
    Does the same as \cs{ac} but uppercases the first letter -- this may be
    needed at the beginning of a sentence.
  \command{acs}[\marg{id}]
    Prints the short form of the acronym \meta{id}.
  \command{Acs}[\marg{id}]
    Does the same as \cs{acs} but uppercases the first letter.
  \command{acl}[\marg{id}]
    Prints the long form of the acronym \meta{id}.
  \command{Acl}[\marg{id}]
    Does the same as \cs{acl} but uppercases the first letter.
  \command{acf}[\marg{id}]
    Prints the full form of the acronym \meta{id}.
  \command{Acf}[\marg{id}]
    Does the same as \cs{acf} but uppercases the first letter.
\end{commands}
Let's say you defined \acs*{cd} as follows:
\begin{sourcecode}
  \DeclareAcronym{cd}{
    short = CD ,
    long  = compact disc
  }
\end{sourcecode}
Then the usage is
\begin{example}[side-by-side]
  \begin{tabular}{ll}
    first  & \ac{cd} \\
    second & \ac{cd} \\
    long   & \acl{cd} \\
    short  & \acs{cd} \\
    full   & \acf{cd}
  \end{tabular}
\end{example}

\part{Comprehensive details}

\clearpage
\appendix
\part{Appendix}

\printacronyms[
  preamble = {Below all abbreviations are listed which have been defined for
    the manual.} ,
  exclude = exclude
]

% \acshow{ecu}

% \printbibliography

\end{document}
