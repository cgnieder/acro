% arara: pdflatex: { interaction: nonstopmode }
% arara: biber
% arara: pdflatex: { interaction: nonstopmode }
% !arara: pdflatex: { interaction: nonstopmode }
% !arara: pdflatex: { interaction: nonstopmode }
% --------------------------------------------------------------------------
% the ACRO package
% 
%   Typeset Acronyms
% 
% --------------------------------------------------------------------------
% Clemens Niederberger
% Web:    https://github.com/cgnieder/acro/
% E-Mail: contact@mychemistry.eu
% --------------------------------------------------------------------------
% Copyright 2011--2020 Clemens Niederberger
% 
% This work may be distributed and/or modified under the
% conditions of the LaTeX Project Public License, either version 1.3
% of this license or (at your option) any later version.
% The latest version of this license is in
%   http://www.latex-project.org/lppl.txt
% and version 1.3 or later is part of all distributions of LaTeX
% version 2005/12/01 or later.
% 
% This work has nce status `maintained'.
% 
% The Current Maintainer of this work is Clemens Niederberger.
% --------------------------------------------------------------------------
% The acro package consists of the files
% - acro.sty, acro.definitions.tex, acro.cfg
% - acro-manual.tex, acro-manual.pdf, acro-manual.cls
% - acro.history, README
% --------------------------------------------------------------------------
% If you have any ideas, questions, suggestions or bugs to report, please
% feel free to contact me.
% --------------------------------------------------------------------------
\PassOptionsToPackage{ngerman,latin,english}{babel}
\PassOptionsToPackage{version=3,upgrade}{acro}
\documentclass{acro-manual}

\usepackage{todonotes}

\addbibresource{\jobname.bib}
\addbibresource{cnltx.bib}
\begin{filecontents}{\jobname.bib}
@online{wikipedia,
  author   = {Wikipedia},
  title    = {Acronym and initialism},
  urldate  = {2012-06-21},
  url      = {http://en.wikipedia.org/wiki/Acronyms},
  year     = {2012}
}
@online{NewYork,
  author   = {Wikipedia},
  title    = {New York City},
  urldate  = {2012-09-27},
  url      = {http://en.wikipedia.org/wiki/New_York_City},
  year     = {2012}
}
@manual{interface3,
  author    = {{The \LaTeX3 Project Team}} ,
  shorthand = {L3P} ,
  sortname  = {LaTeX3 Project Team} ,
  title     = {The \LaTeX3 Interfaces} ,
  date      = {2015-09-06} ,
  url       = {http://mirrors.ctan.org/macros/latex/contrib/l3kernel/interface3.pdf}
}
\end{filecontents}

% declare acronyms
\DeclareAcronym{cd}{
  short = CD ,
  long  = compact disc
}
\DeclareAcronym{ctan}{
  short     = ctan ,
  long      = Comprehensive \TeX\ Archive Network ,
  short-format = \scshape ,
  pdfstring = CTAN ,
  short-acc = CTAN ,
  first-style = short-long ,
  single-style = short
}
\DeclareAcronym{ecu}{
  short   = ECU ,
  long    = Steuergerät ,
  foreign = Electronic Control Unit ,
  foreign-babel = english ,
  foreign-locale = englisch
}
\DeclareAcronym{eg}{
  short = e.g\acdot ,
  long  = for example ,
  foreign = exempli gratia ,
  foreign-babel  = latin ,
  short-format = \textit ,
  foreign-format = \textit
}
\DeclareAcronym{etc}{
  short = etc\acdot ,
  long = et cetera ,
  format = \textit ,
  first-style = long
}
\DeclareAcronym{id}{
  short        = id ,
  long         = identification string ,
  short-format = \scshape
}
\DeclareAcronym{jpg}{
  short = JPEG ,
  sort  = jpeg ,
  alt   = JPG ,
  long  = Joint Photographic Experts Group
}
\DeclareAcronym{la}{
  short = LA ,
  long = Los Angeles,
  plural = ,
  class = city
}
\DeclareAcronym{lppl}{
  short = lppl ,
  long = \unexpanded{\LaTeX} Project Public License ,
  short-format = \scshape ,
  pdfstring = LPPL ,
  short-acc = LPPL
}
\DeclareAcronym{MP}{
  short = MP ,
  long  = Member of Parliament ,
  plural-form = Members of Parliament ,
  long-possessive-form = Member's of Parliament
}
\DeclareAcronym{nato}{
  short        = nato ,
  long         = Organisation des Nordatlantikvertrags ,
  foreign      = North Atlantic Treaty Organization ,
  foreign-babel  = english ,
  foreign-locale = englisch ,
  short-format = \scshape
}
\DeclareAcronym{ny}{
  short = NY ,
  long = New York ,
  plural = ,
  class = city ,
  cite = NewYork
}
\DeclareAcronym{pdf}{
  short = pdf ,
  long = Portable Document Format ,
  short-format = \scshape ,
  pdfstring = PDF ,
  short-acc = PDF
}
\DeclareAcronym{png}{
  short = PNG ,
  long  = Portable Network Graphics ,
  first-style = short-long ,
  single-style = short
}
\DeclareAcronym{sw}{
  short       = SW ,
  long        = Sammelwerk ,
  long-plural = e ,
  class = exclude
}
\DeclareAcronym{tex.sx}{
  short = \TeX.sx ,
  sort  = TeX.sx ,
  long  = \TeX{} StackExchange
}
\DeclareAcronym{ufo}{
  short   = UFO ,
  long    = unidentified flying object ,
  foreign = unbekanntes Flugobjekt ,
  foreign-plural-form = unbekannte Flugobjekte ,
  foreign-babel = ngerman ,
  long-indefinite = an
}

% because the mannual does not activate the `macros' option:
\newcommand*\nato{\ac{nato}}
\newcommand*\ny{\ac{ny}}

\newcommand*\latin[1]{\textit{#1}}

\begin{document}

\begin{bewareofthedog}
  Hi and thanks that you are testing v3.0 of \acro\ before it is released to
  \ac{ctan}. If you want to test the new version use
  \cs*{usepackage}\Oarg{version=3}\Marg{acro}. With \code{version=2} or no
  option at all you get the old version of acro.  Using
  \cs*{usepackage}\Oarg{version=3,deprecation}\Marg{acro} is supposed to give
  as much meaningful warnings and errors as possible.
\end{bewareofthedog}

\clearpage
\part{Get started with \acro}\label{part:get-started-with}

\section{Licence and requirements}\label{sec:licence-requirements}
\license

\section{\acro\ for the impatient}\label{sec:acro-impatient}

Acronyms are defined in the preamble via the command
\begin{commands}
  \command{DeclareAcronym}[\marg{id}\marg{properties}]
    where \meta{id} is a unique string to identify the acronym and
    \meta{properties} is a key\slash value list of properties.
\end{commands}
In the document acronyms are used with these commands:
\begin{commands}
  \command{ac}[\marg{id}\quad\cs{Ac}\marg{id}]
    \cs{ac} prints the acronym \meta{id}, the first time with full description
    and every subsequent use only the abbreviated form. \cs{Ac} does the same
    but uppercases the first letter -- this may be needed at the beginning of
    a sentence.
  \command{acs}[\marg{id}\quad\cs{Acs}\marg{id}]
    \cs{acs} prints the short form of the acronym \meta{id}. \cs{Acs} does the
    same but uppercases the first letter.
  \command{acl}[\marg{id}\quad\cs{Acl}\marg{id}]
    \cs{acl} prints the long form of the acronym \meta{id}. \cs{Acl} does the
    same but uppercases the first letter.
  \command{acf}[\marg{id}\quad\cs{Acf}\marg{id}]
    \cs{acf} prints the full form of the acronym \meta{id}. \cs{Acf} does the
    same but uppercases the first letter.
\end{commands}
Let's say you defined \acs*{cd} as follows:
\begin{sourcecode}
  \DeclareAcronym{cd}{
    short = CD ,
    long  = compact disc
  }
\end{sourcecode}
Then the usage is
\begin{example}[side-by-side]
  \begin{tabular}{ll}
    first  & \ac{cd} \\
    second & \ac{cd} \\
    long   & \acl{cd} \\
    short  & \acs{cd} \\
    full   & \acf{cd}
  \end{tabular}
\end{example}

\section{Setting options}\label{sec:setting-options}
\subsection{Load-time options}\label{sec:load-time-options}
\acro\ knows only a few set of load-time options which can be used as argument
to \cs*{usepackage}.  To be more precise it knows only one such option:
\begin{options}
  \opt{upgrade}
    When this option is used \acro\ tries to give as much helpful and
    meaningful warning or error messages when a deprecated or removed command
    or setup is used.  This is especially useful if you are upgrading from
    version~2.
\end{options}

\subsection{Setup command}\label{sec:setup-command}
All options of \acro\ that have \emph{not} been mentioned in
section~\ref{sec:load-time-options} have to be set up either with this command
\begin{commands}
  \command{acsetup}[\marg{options}]
    or as option to other commands.  If this is possible then it is described
    when the corresponding commands are explained.  Options usually follow a
    key\slash value syntax like and are always described in the following way:
\end{commands}
\begin{options}
  \opt*{option}
    An option without a value. Those options are very rare if there are any.
  \keyval*{option}{value}\Default{preset}
    An option where a value can be given.  The pre-set value is given to the
    right.
  \keychoice*{option}{\default{choiceA},choiceB,choiceC}\Default{choiceB}
    An option with a determined set of choices. The underlined value is
    chosen if the option is given without value.
  \keybool*{option}
    A boolean option.
  \opt*{option}\Module{module}
    An option at a deeper level belonging to the module \module*{module}.
\end{options}
All of the above is probably clear from an example (using real options):
\begin{sourcecode}
  \acsetup{
    make-links = true , % boolean
    index ,             % boolean
    format = \emph ,    % standard
    list / local ,      % boolean option of the list module
    list/display = all  % choice option of the list module
  }
\end{sourcecode}

\part{Comprehensive description of creation and usage of acronyms}\label{part:compr-descr-creat}
\section{Declaring acronyms and other abbreviations}\label{sec:decl-acronyms-other}

All acronyms have to be declared in the preamble with the following command in
order to be used in the document. Any usage of an acronym which has not been
declared leeds to an error message.
\begin{commands}
  \command{DeclareAcronym}[\marg{id}\marg{list of properties}]
    The basic command for declaring an acronym where \meta{id} is a unique
    string identifying the acronym.  Per default behaviour this is case
    sensitive which means \code{id} is different from \code{ID}, for example.
    There is an option \option{case-sensitive} to change this.
\end{commands}
This command understands a number of properties which are listed in the
following sections.  This is a comprehensive overview over the existing
properties. Most of these properties are explained in more detail in later
sections of this manual.

\begin{bewareofthedog}
  In its simplest form an acronym needs a short and a long form.  Please note
  that both properties \emph{must} be set.
\end{bewareofthedog}

\subsection{Basic properties}\label{sec:basic-properties}
\begin{properties}
  %% short
  \propval{short}{text}\Default!
    The short form of the acronym.  \emph{This property is required}: an
    acronym must have a short form.
\end{properties}
Maybe you mostly have simple acronyms where the \acs{id} and short form are
the same.  In that case you can use
\begin{options}
  \keybool{use-id-as-short}\Default{false}
    to use the \acs{id} of the acronym as short form. For more complicated
    cases this would still allow you to set the short form.
\end{options}
\begin{properties}
  %% long
  \propval{long}{text}\Default!
    The long form of the acronym.  \emph{This property is required}: an
    acronym must have a description.
  %% alt
  \propval{alt}{text}\Default
    Alternative short form.
  %% extra
  \propval{extra}{text}\Default
    Extra information to be added in the list of acronyms.
  %% foreign
  \propval{foreign}{long form in foreign language}\Default
    Can be useful when dealing with acronyms in foreign languages, see
    section~\vref{sec:fore-lang-acronyms} for details.
  %% post
  \propval{post}{text}\Default
    \meta{text} is appended to the acronym in the text but not in the list of
    acronyms.
  %% single
  \propval{single}{text}\Default={long}
    If provided \meta{text} will be used instead of the long form if the
    acronym is only used a single time \emph{and} the option
    \option{single} has been set, see section~\vref{sec:single-appe-an}.
  %% sort
  \propval{sort}{text}\Default={short}
    If used the acronym will be sorted according to this property instead of
    its \acs{id}.
  %% class
  \propval{class}{csv list}\Default
    The class(es) the acronym belongs to.
  %% cite
  \proplit-{cite}{\oarg{prenote}\oarg{postnote}\marg{citation keys}}\Default
    A citation that is printed to the acronym according to an option explained
    later.
  %% index
  \propval{index}{text}Default
    This property allows to overwrite the automatic index entry with an
    arbitrary one.  See section~\vref{sec:indexing} for details.
\end{properties}

\subsection{Properties related to plural and indefinite forms}\label{sec:prop-relat-plur}
\begin{properties}
  %% short-plural
  \propval{short-plural}{text}\Default{s}
    The plural ending appended to the short form.
  %% short-plural-form
  \propval{short-plural-form}{text}\Default
    The plural short form of the acronym; replaces the short form when used
    instead of appending the plural ending.
  %% long-plural
  \propval{long-plural}{text}\Default{s}
    The plural ending appended to the long form.
  %% long-plural-form
  \propval{long-plural-form}{text}\Default
    Plural long form of the acronym; replaces the long form when used
    instead of appending the plural ending.
   %% alt-plural
  \propval{alt-plural}{text}\Default{s}
    The plural ending appended to the alternative form.
  %% alt-plural-form
  \propval{alt-plural-form}{text}\Default
    The plural alternative form of the acronym; replaces the alternative form
    when used instead of appending the plural ending.
  %% foreign-plural
  \propval{foreign-plural}{text}\Default{s}
    The plural ending appended to the foreign form.
  %% foreign-plural-form
  \propval{foreign-plural-form}{text}\Default
    Plural foreign form of the acronym; replaces the foreign form when used
    instead of appending the plural ending.
  %% short-indefinite
  \propval{short-indefinite}{text}\Default{a}
    Indefinite article for the short form.
  %% long-indefinite
  \propval{long-indefinite}{text}\Default{a}
    Indefinite article for the long form.
  %% alt-indefinite
  \propval{alt-indefinite}{text}\Default{a}
    Indefinite article for the alternative form.
\end{properties}

\subsection{Properties related to formatting}\label{sec:prop-relat-form}
\begin{properties}
  \propval{format}{\TeX{} code}\Default
    The format used for both short and long form of the acronym.
  %% short-format
  \propval{short-format}{\TeX{} code}\Default={format}
    The format used for the short form of the acronym.
  %% long-format
  \propval{long-format}{\TeX{} code}\Default={format}
    The format used for the long form of the acronym.
  %% alt-format
  \propval{alt-format}{\TeX{} code}\Default={short-format}
    The format used for the alternative form of the acronym. If this is not
    given the short format will be used.
  %% extra-format
  \propval{extra-format}{\TeX{} code}\Default
    The format used for the additional information of the acronym.
  %% foreign-format
  \propval{foreign-format}{\TeX{} code}\Default
    The format used for the foreign form of the acronym.
  %% single-format
  \propval{single-format}{\TeX{} code}\Default={long-format}
    The format used for the acronym if the acronym is only used a single
    time.
  %% first-style
  \propchoice{first-style}{long-short,short-long,short,long,footnote}\Default
    The style of the first appearance of the acronym, see also
    section~\vref{sec:first-or-full}.
  %% single-style
  \propchoice{single-style}{long-short,short-long,short,long,footnote}\Default
    The style of a single appearance of the acronym, see also
    section~\vref{sec:single-appe-an}.
\end{properties}

\subsection{Properties related to the created \acs*{pdf} file}\label{sec:prop-relat-creat}
\begin{properties}
  %% pdfstring
  \propval{pdfstring}{pdfstring}\Default={short}
    Used as \acs{pdf} string replacement in bookmarks when used together with
    the \pkg{hyperref}~\cite{pkg:hyperref} or the \pkg{bookmark}
    package~\cite{pkg:bookmark}.
  %% pdfcomment
  \propval{pdfcomment}{text}
    Sets a tooltip description for an acronym.  For actually getting
    tooltips you also need an appropriate setting of the options
    \option{pdfcomment/cmd} and \option{pdfcomment/use}, see also
    section~\vref{sec:pdf-comments}.
  %% short-acc
  \propval{short-acc}{text}\Default={short}
    Sets the \code{ActualText} property as presented by the \pkg{accsupp}
    package for the short form of the acronym.
  %% long-acc
  \propval{long-acc}{text}\Default={long}
    Sets the \code{ActualText} property as presented by the \pkg{accsupp}
    package for the long form of the acronym.
  %% alt-acc
  \propval{alt-acc}{text}\Default={alt}
    Sets the \code{ActualText} property as presented by the \pkg{accsupp}
    package for the alternative short form of the acronym.
  %% foreign-acc
  \propval{foreign-acc}{text}\Default={foreign}
    Sets the \code{ActualText} property as presented by the \pkg{accsupp}
    package for the foreign form of the acronym.
  %% short-acc
  \propval{extra-acc}{text}\Default={extra}
    Sets the \code{ActualText} property as presented by the \pkg{accsupp}
    package for the extra information of the acronym.
  %% single-acc
  \propval{single-acc}{text}\Default={long-acc}
    Sets the \code{ActualText} property as presented by the \pkg{accsupp}
    package for a single appearance of the acronym.
\end{properties}

\subsection{Futher properties}\label{sec:futher-properties}
\begin{properties}
  %% list
  \propval{list}{text}\Default={long}
    If specified this will be written in the list as description instead of
    the long form if the corresponding list template supports it.
  %% foreign-babel
  \propval{foreign-babel}{language}\Default
    The \pkg{babel}~\cite{pkg:babel} or
    \pkg{polyglossia}~\cite{pkg:polyglossia} language of the foreign form.
    This language is used to wrap the entry with
    \cs*{foreignlanguage}\marg{language} if either \pkg{babel} or
    \pkg{polyglossia} is loaded.  You'll need to take care that the
    corresponding language is loaded by \pkg{babel} or \pkg{polyglossia}.
  %% foreign-locale
  \propval{foreign-locale}{language}\Default
    The language name that is output when the option
    \module{locale}\code{/}\option{display} is used.  If this property is not
    set then the appropriate value might be derived from
    \property{foreign-babel}. See section~\vref{sec:fore-lang-acronyms} for
    details.
  %% index-sort
  \propval{index-sort}{text}\Default={sort}
    If you use the option \option{index} every occurrence of an acronym is
    recorded to the index and sorted by its \acs{id} or (if set) by the value
    of the \property{sort} property.  This property allows to set an
    individual sorting option for the index.  See section~\vref{sec:indexing}
    for details
\end{properties}

\section{Using acronyms}\label{sec:using-acronyms}
There are a number of commands to use acronyms with. Their names always follow
the same pattern which should make their usage intuitive immediately.

All of these commands have a starred form which means \enquote{don't count
  this as usage}. All of these commands also have an optional argument that
allows to set options for that usage only.
\begin{commands}
  \command*{acrocommand}[\sarg\oarg{options}\marg{id}]
    This is the general syntax of all of the commands listed below.  The star
    and the optional argument is left way for the sake of readability.
  \command{ac}[\marg{id} \cs{Ac}\marg{id} \cs{acp}\marg{id} \cs{Acp}\marg{id}
  \cs{iac}\marg{id} \cs{Iac}\marg{id}]
    \cs{ac} prints the acronym \meta{id}, the first time with full description
    and every subsequent use only the abbreviated form. \cs{Ac} does the same
    but uppercases the first letter -- this may be needed at the beginning of
    a sentence.  The commands \cs{acp} and \cs{Acp}, resp., print the
    corresponding plural forms.  The commands \cs{iac} and \cs{Iac}, resp.,
    print indefinite forms.
  \command{acs}[\marg{id} \cs{Acs}\marg{id} \cs{acsp}\marg{id} \cs{Acsp}\marg{id}
  \cs{iacs}\marg{id} \cs{Iacs}\marg{id}]
    \cs{acs} prints the short form of the acronym \meta{id}. \cs{Acs} does the
    same but uppercases the first letter. The commands \cs{acsp} and
    \cs{Acsp}, resp., print the corresponding plural forms. The commands
    \cs{iacs} and \cs{Iacs}, resp., print indefinite forms.
  \command{acl}[\marg{id} \cs{Acl}\marg{id} \cs{aclp}\marg{id} \cs{Aclp}\marg{id}
  \cs{iacl}\marg{id} \cs{Iacl}\marg{id}]
    \cs{acl} prints the long form of the acronym \meta{id}. \cs{Acl} does the
    same but uppercases the first letter. The commands \cs{aclp} and
    \cs{Aclp}, resp., print the corresponding plural forms. The commands
    \cs{iacl} and \cs{Iacl}, resp., print indefinite forms.
  \command{aca}[\marg{id} \cs{Aca}\marg{id} \cs{acap}\marg{id} \cs{Acap}\marg{id}
  \cs{iaca}\marg{id} \cs{Iaca}\marg{id}]
    \cs{aca} prints the alternative short form of the acronym \meta{id}.
    \cs{Aca} does the same but uppercases the first letter. The commands
    \cs{acap} and \cs{Acap}, resp., print the corresponding plural forms. The
    commands \cs{iaca} and \cs{Iaca}, resp., print indefinite forms.
  \command{acf}[\marg{id} \cs{Acf}\marg{id} \cs{acfp}\marg{id} \cs{Acfp}\marg{id}
  \cs{iacf}\marg{id} \cs{Iacf}\marg{id}]
    \cs{acf} prints the full form of the acronym \meta{id}. \cs{Acf} does the
    same but uppercases the first letter. The commands \cs{acfp} and
    \cs{Acfp}, resp., print the corresponding plural forms. The commands
    \cs{iacf} and \cs{Iacf}, resp., print indefinite forms.
\end{commands}

The usage should be clear. Let's assume you have defined an acronym
\acs*{ufo}\label{ufo} like this:
\begin{sourcecode}
  \DeclareAcronym{ufo}{
    short = UFO ,
    long = unidentified flying object ,
    foreign = unbekanntes Flugobjekt ,
    foreign-plural-form = unbekannte Flugobjekte ,
    foreign-babel = ngerman ,
    long-indefinite = an
  }
\end{sourcecode}
Then typical outputs look like this:
\begin{example}
  \ac{ufo} \\
  \iac{ufo} \\
  \iacl{ufo} \\
  \Iacf{ufo} \\
  \acfp{ufo}
\end{example}

\section{Alternative short forms}\label{sec:altern-short-forms}
Sometimes expressions have two different short forms. And example might be
\acs*{jpg} which also often is \aca*{jpg}. This is what the property
\property{alt} is there for.
\begin{properties}
  \propval{alt}{text}
    Alternative short form.
\end{properties}
Let's define \acs*{jpg}:
\begin{sourcecode}
  \DeclareAcronym{jpg}{
    short = JPEG ,
    sort  = jpeg ,
    alt   = JPG ,
    long  = Joint Photographic Experts Group
  }
\end{sourcecode}
And let's see how to use it:
\begin{example}
  \ac{jpg} \\
  \ac{jpg} \\
  \aca{jpg}
\end{example}
As you can see the full form shows both short forms of the acronym. This could
be changed by altering the template for the full form, see
section~\vref{sec:templates} and section~\vref{sec:first-or-full}.  The
alternative form is also printed in the list of acronyms, see
section~\vref{sec:acronyms}.  This can also be changed by altering the
template for the list, again see section~\ref{sec:templates}.

\section{The first or full appearance}\label{sec:first-or-full}
If an acronym is used for the first time with \cs{ac} (after any number of
usages with the starred forms of the usage commands listed in
section~\vref{sec:using-acronyms}) or if an acronym is used \cs{acf}, then the
first or full appearance of the acronym is printed\footnote{This usually
  requires at least two compilations.}.

The first or full appearance of an acronym is determined by this option:
\begin{options}
  \keychoice{first-style}{long-short,short-long,short,long,footnote}\Default{long-short}
    The style of the first appearance of the acronym. This options sets the
    appearance for all acronyms.  Available options in reality are the names
    of all defined templates of the type \code{acronym}. All pre-defined
    templates can be found in section~\vref{sec:pre-defin-templ}.
\end{options}
It might be desirable to set the first appearance of an acronym
individually. This is possible by setting the corresponding property:
\begin{properties}
  \propchoice{first-style}{long-short,short-long,short,long,footnote}\Default
    The style of the first appearance of the acronym.
\end{properties}
Let's again look at an example:
\begin{example}[side-by-side]
  \acf[first-style=long-short]{cd} \\
  \acf[first-style=short-long]{cd} \\
  \acf[first-style=footnote]{cd} \\
  \acf[first-style=long]{cd} \\
  \acf[first-style=short]{cd}
\end{example}
This also demonstrates the use of the optional argument.

An example of an abbreviation that should have \code{long} as first appearance
might be \enquote{\acs*{etc}}, defined like this
\begin{sourcecode}
  \DeclareAcronym{etc}{
    short = etc\acdot ,
    long = et cetera ,
    format = \textit ,
    first-style = long
  }
\end{sourcecode}
and output like this:
\begin{example}[side-by-side]
  \ac{etc}, \ac{etc} \ac{etc}.
\end{example}
The command \cs{acdot} is explained in section~\vref{sec:trailing-tokens}.
Basically it checks if a dot follows and outputs a dot if not.

\section{Single appearances of an acronym}\label{sec:single-appe-an}
If an acronym is used only once (not counting usages with the starred forms of
the usage commands listed in section~\vref{sec:using-acronyms}), then the
single appearance of the acronym is printed\footnote{This usually requires at
  least two compilations.}.

The single appearance of an acronym is determined by this option:
\begin{options}
  \keychoice{single}{\default{true},false,\meta{number}}\Default{false}
    This option determines wether a single appearance of an acronym counts as
    \emph{usage}. It might be desirable in such cases that an acronym is
    simply printed as long form and not added to the list of acronym.  This is
    what this option does.  With \meta{number} the minimal number of usages
    can be given that needs to be exceeded.  \keyis{single}{1} is the same as
    \keyis{single}{true}.
  \keychoice{single-style}{long-short,short-long,short,long,footnote}\Default{long}
    The style of the single appearance of an acronym.  Can be used to
    determine how a single appearance is printed if the option \option{single}
    has been set. This options sets the appearance for all acronyms.
    Available options in reality are the names of all defined templates of the
    type \code{acronym}. All pre-defined templates can be found in
    section~\vref{sec:pre-defin-templ}.
\end{options}

If you like you can also set the single appearance of an acronym individually:
\begin{properties}
  \propval{single}{text}\Default={long}
    If provided \meta{text} will be used instead of the long form if the
    acronym is only used a single time \emph{and} the option
    \option{single} has been set.
  \propval{single-format}{\TeX{} code}\Default={long-format}
    The format used for the acronym if the acronym is only used a single
    time.
  \propchoice{single-style}{long-short,short-long,short,long,footnote}\Default
    The style of the single appearance of the acronym.
\end{properties}
Let's again look at an example. The acronym \acs*{png} is defined as follows:
\begin{sourcecode}
  \DeclareAcronym{png}{
    short = PNG ,
    long  = Portable Network Graphics ,
    first-style = short-long ,
    single-style = short
  }
\end{sourcecode}
And it is used only once in this manual\footnote{You will find it in the list
  of acronyms in section~\ref{sec:acronyms} nonetheless as this document does
  \keyis{list/display}{all}.}:
\begin{example}[side-by-side]
  \ac{png}
\end{example}  
Please be aware that \cs{acf} would still print the full form, of course.

\section{Plural forms and other endings}\label{sec:plural-forms-other}
\subsection{The plural ending and the plural form}\label{sec:plural-ending-form}
Not in all languages plural forms are as easy as always appending an
\enquote{s}.  Not even English.  Sometimes there's other endings
instead\footnote{German is full of such examples.}.  This is why \acro\ has
quite a number of different properties related to plural forms or endings:
\begin{properties}
  %% short-plural
  \propval{short-plural}{text}\Default{s}
    The plural ending appended to the short form.
  %% short-plural-form
  \propval{short-plural-form}{text}\Default
    The plural short form of the acronym; replaces the short form when used
    instead of appending the plural ending.
  %% long-plural
  \propval{long-plural}{text}\Default{s}
    The plural ending appended to the long form.
  %% long-plural-form
  \propval{long-plural-form}{text}\Default
    Plural long form of the acronym; replaces the long form when used
    instead of appending the plural ending.
   %% alt-plural
  \propval{alt-plural}{text}\Default{s}
    The plural ending appended to the alternative form.
  %% alt-plural-form
  \propval{alt-plural-form}{text}\Default
    The plural alternative form of the acronym; replaces the alternative form
    when used instead of appending the plural ending.
  %% foreign-plural
  \propval{foreign-plural}{text}\Default{s}
    The plural ending appended to the foreign form.
  %% foreign-plural-form
  \propval{foreign-plural-form}{text}\Default
    Plural foreign form of the acronym; replaces the foreign form when used
    instead of appending the plural ending.  
\end{properties}
There are two options which allow to change the default values for the whole
document:
\begin{options}
  \keyval{short-plural-ending}{text}\Default{s}
    Defines the plural ending for the short forms to be \meta{text}.
  \keyval{long-plural-ending}{text}\Default{s}
    Defines the plural ending for the long forms to be \meta{text}.
\end{options}
Now let's see two simple examples demonstrating the two different kinds of plural
settings:
\begin{sourcecode}
  \DeclareAcronym{sw}{
    short = SW ,
    long = Sammelwerk ,
    long-plural = e
  }
  \DeclareAcronym{MP}{
    short = MP ,
    long  = Member of Parliament ,
    plural-form = Members of Parliament
  }
\end{sourcecode}
The first one has another plural ending than the usual \enquote{s}. The second
one has a different plural form altogether because appending an \enquote{s}
would give a wrong form:
\begin{example}[side-by-side]
  \acfp{sw} \par
  \acfp{MP}
\end{example}

\subsection{Other endings}\label{sec:other-endings}
There are other such concepts which is why \acro\ generalizes the concept of
endings.
\begin{commands}
  \command{DeclareAcroEnding}[\marg{name}\marg{short default}\marg{long
    default}]
    This command can be used to define properties and options analoguous to
    the plural endings which have been defined this way:
\end{commands}
\begin{sourcecode}
  \DeclareAcroEnding{plural}{s}{s}
\end{sourcecode}
In general \cs{DeclareAcroEnding}\marg{foo}\marg{x}\marg{y} defines these
options
\begin{options}
  \keyval*{short-\meta{foo}-ending}{value}\Default*{\meta{x}}
  \keyval*{long-\meta{foo}-ending}{value}\Default*{\meta{y}}
\end{options}
and these properties
\begin{properties}
  \propval*{short-\meta{foo}}{value}\Default*{\meta{x}}
  \propval*{short-\meta{foo}-form}{value}\Default*
  \propval*{alt-\meta{foo}}{value}\Default*{\meta{x}}
  \propval*{alt-\meta{foo}-form}{value}\Default*
  \propval*{long-\meta{foo}}{value}\Default*{\meta{y}}
  \propval*{long-\meta{foo}-form}{value}\Default*
  \propval*{foreign-\meta{foo}}{value}\Default*{\meta{y}}
  \propval*{foreign-\meta{foo}-form}{value}\Default*
  \propval*{single-\meta{foo}}{value}\Default*{\meta{y}}
  \propval*{single-\meta{foo}-form}{value}\Default*
  \propval*{extra-\meta{foo}}{value}\Default*{\meta{y}}
  \propval*{extra-\meta{foo}-form}{value}\Default*
\end{properties}
In addition another command is defined which is meant to be used in
template definitions.
\begin{commands}
  \command*{acro\meta{foo}}
    This command tells the template that the ending \meta{foo} should be
    used.
\end{commands}
Section~\vref{sec:own-acronym-commands} has an example of how this can be used
to define a possessive ending and commands that make use of them like this:
\begin{example}[side-by-side]
  \acfg{MP}
\end{example}

\section{Indefinite forms}\label{sec:indefinite-forms}
Indefinite forms can be a problem if the short and the long form of acronyms
have different indefinite articles\footnote{This may very well be a language
  specific issue.}.
\begin{example}[side-by-side]
  \acreset{ufo}%
  a \ac{ufo} \par
  an \ac{ufo}
\end{example}
And what good would it be to uase a package like \acro\ if you have to keep
track of of and second uses, anyway?  This is why \acs{ufo} should be defined
like we did on page~\pageref{ufo}.  We then can just use the dedicated
commands and let them decide for us:
\begin{example}[side-by-side]
  \acreset{ufo}%
  \iac{ufo} \par
  \iac{ufo}
\end{example}
The commands which also output the indefinite article all start with an
\enquote{i} and have all been described in section~\vref{sec:using-acronyms}
already: \cs{iac}, \cs{Iac}, \cs{iacs}, \cs{Iacs}, \cs{iacl}, \cs{Iacl},
\cs{iaca}, \cs{Iaca}, \cs{iacf}, and \cs{Iacf}.

\section{Foreign language acronyms}\label{sec:fore-lang-acronyms}
Sometimes and in some fields more often than in others abbreviations are used
that are derived from another language.  \acro\ provides a number of
properties for such cases:
\begin{properties}
  %% foreign
  \propval{foreign}{long form in foreign language}\Default
    Can be useful when dealing with acronyms in foreign languages, see
    section~\vref{sec:fore-lang-acronyms} for details.
  %% foreign-plural
  \propval{foreign-plural}{text}\Default{s}
    The plural ending appended to the foreign form.
  %% foreign-plural-form
  \propval{foreign-plural-form}{text}\Default
    Plural foreign form of the acronym; replaces the foreign form when used
    instead of appending the plural ending.
  %% foreign-format
  \propval{foreign-format}{\TeX{} code}\Default
    The format used for the foreign form of the acronym.
  %% foreign-babel
  \propval{foreign-babel}{language}\Default
    The \pkg{babel} or \pkg{polyglossia} language of the foreign form. This
    language is used to wrap the entry with
    \cs*{foreignlanguage}\marg{language} if either \pkg{babel} or
    \pkg{polyglossia} is loaded.  You'll need to take care that the
    corresponding language is loaded by \pkg{babel} or \pkg{polyglossia}.
  %% foreign-locale
  \propval{foreign-locale}{language}\Default
    The language name that is output when the option
    \module{locale}\code{/}\option{display} is used.  If this property is not
    set then the appropriate value might be derived from
    \property{foreign-babel}.
\end{properties}
There are also some options:
\begin{options}
  \keybool{display}\Module{locale}\Default{false}
    This options determines wether the language of the foreign form is printed
    or not when the full form of the acronym is printed.
  \keybool{display}\Module{list/locale}\Default{false}
    The same but for the list of acronyms.
  \keyval{format}{code}\Module{locale}\Default{\cs*{em}\cs*{text\_titlecase\_first:n}}
    Determines how said language is formatted when printed.  The last command
    in \meta{code} may take a mandatory argument.
\end{options}

Let's say you are writing a German document and are using the abbreviation
\acs*{ecu} for \acl*{ecu} which stems from the English \enquote{Electronic
  Control Unit}.  Then you can define it as follows:
\begin{sourcecode}
  \DeclareAcronym{ecu}{
    short   = ECU ,
    long    = Steuergerät ,
    foreign = Electronic Control Unit ,
    foreign-babel = english ,
    foreign-locale = englisch
  }
\end{sourcecode}
Now the abbreviation is introduced so that everyone understands the confusion:
\begin{example}
  \ac{ecu} \par
  \acsetup{locale/display,locale/format=\emph}
  \acf{ecu}
\end{example}

The property \property{foreign-babel} is used for ensuring correct hyphenation
as long as you use \pkg{babel} or \pkg{polyglossia} and load the corresponding
language, too.  If you are writing your document in English then \acro\ is
able to deduce the language used for the \enquote{locale} field by itself:
\begin{sourcecode}
  \DeclareAcronym{eg}{
    short = e.g\acdot ,
    long  = for example ,
    foreign = exempli gratia ,
    foreign-babel  = latin ,
    short-format = \textit ,
    foreign-format = \textit
  }
\end{sourcecode}
\begin{example}
  \acsetup{locale/display,first-style=short-long}
  \acf{eg}
\end{example}

\section{Uppercasing}\label{sec:uppercasing}

\section{Printing the list}\label{sec:printing-list}
\subsection{The main command and its options}\label{sec:main-list}
\subsection{Several lists using classes}\label{sec:lists-classes}
\subsection{Local lists}\label{sec:local-lists}

\section{Trailing tokens}\label{sec:trailing-tokens}

\section{Citing and indexing}\label{sec:citing-indexing}
\subsection{Citing}\label{sec:citing}
\subsection{Indexing}\label{sec:indexing}

\section{Using or resetting acronyms}\label{sec:using-or-resetting}

\section{Localisation}\label{sec:localisation}

\section{Bookmarks and accessibility support}\label{sec:bookm-access-supp}
\subsection{\pkg*{hyperref} support}\label{sec:hyperref-support}
\subsection{\acs*{pdf} comments}\label{sec:pdf-comments}
\subsection{Accessibility support}\label{sec:access-supp}

\part{Extending \acro}\label{part:extending-acro}
\section{Templates}\label{sec:templates}
\subsection{Pre-defined templates}\label{sec:pre-defin-templ}
\subsection{Defining new templates}\label{sec:defin-new-templ}
\subsection{New acronym templates}\label{sec:new-acronym-templ}
\subsection{New list templates}\label{sec:new-list-templates}
\subsection{New heading templates}\label{sec:new-head-templ}


\section{Own acronym commands}\label{sec:own-acronym-commands}

\begin{sourcecode}
  \DeclareAcroEnding{possessive}{'s}{'s}

  \NewAcroCommand\acg{m}{\acropossessive\AcroUseTemplate{first}{#1}}
  \NewAcroCommand\acsg{m}{\acropossessive\AcroUseTemplate{short}{#1}}
  \NewAcroCommand\aclg{m}{\acropossessive\AcroUseTemplate{long}{#1}}
  \NewAcroCommand\acfg{m}{%
    \acrofull
    \acropossessive
    \AcroUseTemplate{first}{#1}%
  }
  \NewAcroCommand\iacsg{m}{%
    \acroindefinite
    \acropossessive
    \AcroUseTemplate{short}{#1}%
  }
\end{sourcecode}

\clearpage
\appendix
\part{Appendix}\label{part:appendix}

\section{Examples}\label{sec:examples}
\todo{}

\printacronyms[
  preamble = {\label{sec:acronyms}Below all abbreviations are listed which
    have been defined for the manual.} ,
  exclude = exclude
]

\printbibliography

\end{document}
