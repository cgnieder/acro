% --------------------------------------------------------------------------
% the ACRO package
% 
%   Typeset Acronyms
% 
% --------------------------------------------------------------------------
% Clemens Niederberger
% Web:    https://github.com/cgnieder/acro/
% E-Mail: contact@mychemistry.eu
% --------------------------------------------------------------------------
% Copyright 2011--2020 Clemens Niederberger
% 
% This work may be distributed and/or modified under the
% conditions of the LaTeX Project Public License, either version 1.3c
% of this license or (at your option) any later version.
% The latest version of this license is in
%   http://www.latex-project.org/lppl.txt
% and version 1.3c or later is part of all distributions of LaTeX
% version 2008/05/04 or later.
% 
% This work has the LPPL maintenance status `maintained'.
% 
% The Current Maintainer of this work is Clemens Niederberger.
% --------------------------------------------------------------------------
\PassOptionsToPackage{ngerman,latin,english}{babel}
\PassOptionsToPackage{version=3,upgrade}{acro}
\documentclass{acro-manual}

\begin{filecontents}{\jobname.bib}
@online{NewYork,
  author   = {Wikipedia},
  title    = {New York City},
  urldate  = {2020-04-11},
  url      = {http://en.wikipedia.org/wiki/New_York_City},
  year     = {2020}
}
\end{filecontents}

\addbibresource{\jobname.bib}
\addbibresource{cnltx.bib}

% declare example acronyms
\DeclareAcronym{cd}{
  short = CD ,
  long  = compact disc
}
\DeclareAcronym{ctan}{
  short     = ctan ,
  long      = Comprehensive \TeX\ Archive Network ,
  short-format = \scshape ,
  pdfstring = CTAN ,
  short-acc = CTAN ,
  first-style = short-long ,
  single-style = short
}
\DeclareAcronym{ecu}{
  short   = ECU ,
  long    = Steuergerät ,
  foreign = Electronic Control Unit ,
  foreign-babel = english ,
  foreign-locale = englisch
}
\DeclareAcronym{eg}{
  short = e.g\acdot ,
  long  = for example ,
  foreign = exempli gratia ,
  foreign-babel  = latin ,
  short-format = \textit ,
  foreign-format = \textit
}
\DeclareAcronym{etc}{
  short = etc\acdot ,
  long = et cetera ,
  format = \textit ,
  first-style = long ,
  plural = 
}
\DeclareAcronym{hadopi}{
  short = HADOPI ,
  long = Haute Autorit\'{e} pour la diffusion des \oe uvres et la protection des
    droits sur l’Internet ,
  short-definite = l'\nospace ,
  long-definite = la ,
}
\DeclareAcronym{id}{
  short        = id ,
  long         = identification string ,
  short-format = \scshape ,
  short-acc    = ID ,
  pdfstring    = ID ,
  first-style  = short
}
\DeclareAcronym{jpg}{
  short = JPEG ,
  sort  = jpeg ,
  alt   = JPG ,
  long  = Joint Photographic Experts Group
}
\DeclareAcronym{la}{
  short = LA ,
  long = Los Angeles,
  plural = ,
  tag = city
}
\DeclareAcronym{lppl}{
  short = lppl ,
  long = \unexpanded{\LaTeX} Project Public License ,
  short-format = \scshape ,
  pdfstring = LPPL ,
  short-acc = LPPL
}
\DeclareAcronym{MP}{
  short = MP ,
  long  = Member of Parliament ,
  plural-form = Members of Parliament ,
  long-possessive-form = Member's of Parliament
}
\DeclareAcronym{nato}{
  short        = nato ,
  long         = Organisation des Nordatlantikvertrags ,
  foreign      = North Atlantic Treaty Organization ,
  foreign-babel  = english ,
  foreign-locale = englisch ,
  short-format = \scshape
}
\DeclareAcronym{ny}{
  short = NY ,
  long = New York ,
  plural = ,
  tag = city ,
  cite = NewYork
}
\DeclareAcronym{pdf}{
  short = pdf ,
  long = Portable Document Format ,
  short-format = \scshape ,
  pdfstring = PDF ,
  short-acc = PDF
}
\DeclareAcronym{png}{
  short = PNG ,
  long  = Portable Network Graphics ,
  first-style = short-long ,
  single-style = short
}
\DeclareAcronym{PU}{
  short = PU ,
  long = Polyurethan ,
  long-plural = e
}
\DeclareAcronym{sw}{
  short       = SW ,
  long        = Sammelwerk ,
  long-plural = e ,
  tag = exclude
}
\DeclareAcronym{tex.sx}{
  short = \TeX.sx ,
  sort  = TeX.sx ,
  long  = \TeX{} StackExchange
}
\DeclareAcronym{ufo}{
  short = UFO ,
  long = unidentified flying object ,
  foreign = unbekanntes Flugobjekt ,
  foreign-plural-form = unbekannte Flugobjekte ,
  foreign-acc-plural-form = unbekannte Flugobjekte ,
  foreign-babel = ngerman ,
  long-indefinite = an
}

% declare glossary terms:
\DeclareAcronym{property}{
  long = A \property*{property} is an option to the second argument of the
    \cs*{DeclareAcroym} command.  They are options of an individual acronym if
    you will. ,
  tag = glossary , no-index
}
\DeclareAcronym{option}{
  long = {An \option*{option} is a package option of \acro\ which must set
    with \cs*{acsetup}. It \emph{cannot} be set as option to \cs*{usepackage}.
    Options usually also can be set in the optional argument of \cs*{ac} and
    friends.} ,
  tag = glossary , no-index
}
\DeclareAcronym{load-time option}{
  long = {A load-time \option*{option} is a package option of \acro\ which
    \emph{must} be set as option to
    \cs*{usepackage}\oarg{options}\Marg{acro}.} ,
  tag = glossary , no-index
}
\DeclareAcronym{template}{
  long = {A template determines how different objects of \acro\ are printed.
    This includes the acronyms themselves but also for example the list of
    acronyms as a whole.} ,
  tag = glossary , no-index
}
\DeclareAcronym{articles}{
  long = {Articles are prefixes to acronyms, usually separated with a blank.
    \emph{Different types of articles are mutually exclusive.}} ,
  tag = glossary , no-index
}
\DeclareAcronym{endings}{
  long = {Endings are postfixes to acronyms, usually not separated from the
    acronym. \emph{Different types of endings are mutually exclusive.}} ,
  tag = glossary , no-index
}
\DeclareAcronym{translations}{
  long = {Localisation strings which can be modified.} ,
  tag = glossary , no-index
}

\newcommand*\issues{\url{https://github.com/cgnieder/acro/issues}}
\newcommand*\latin[1]{\textit{#1}}

\begin{document}

\clearpage
\part{Get started with \acro}\label{part:get-started-with}

\section{Licence and requirements}\label{sec:licence-requirements}
\license

\section{Glossary}
\printacronyms[include=glossary,template=glossary]

\section{\acro\ for the impatient}\label{sec:acro-impatient}

Acronyms are defined in the preamble via the command
\begin{commands}
  \command{DeclareAcronym}[\marg{id}\marg{properties}]
    where \meta{id} is a unique string to identify the acronym and
    \meta{properties} is a key\slash value list of acronym properties.  These
    include:
\end{commands}
\begin{properties}
  %% short
  \propval{short}{text}\Default!
    The short form of the acronym.  \emph{This property is required}: an
    acronym must have a short form.
  %% long
  \propval{long}{text}\Default!
    The long form of the acronym.  \emph{This property is required}: an
    acronym must have a description.
\end{properties}
\begin{bewareofthedog}
  In its simplest form an acronym needs a short and a long form.  Please note
  that both properties \emph{must} be set.
\end{bewareofthedog}  

In the document acronyms are used with these commands:
\begin{commands}
  \command{ac}[\marg{id}\quad\cs{Ac}\marg{id}]
    \cs{ac} prints the acronym \meta{id}, the first time with full description
    and every subsequent use only the abbreviated form. \cs{Ac} does the same
    but uppercases the first letter -- this may be needed at the beginning of
    a sentence.
  \command{acs}[\marg{id}\quad\cs{Acs}\marg{id}]
    \cs{acs} prints the short form of the acronym \meta{id}. \cs{Acs} does the
    same but uppercases the first letter.
  \command{acl}[\marg{id}\quad\cs{Acl}\marg{id}]
    \cs{acl} prints the long form of the acronym \meta{id}. \cs{Acl} does the
    same but uppercases the first letter.
  \command{acf}[\marg{id}\quad\cs{Acf}\marg{id}]
    \cs{acf} prints the full form of the acronym \meta{id}. \cs{Acf} does the
    same but uppercases the first letter.
\end{commands}

Let's say you defined \acs*{cd} as follows:
\begin{sourcecode}
  \DeclareAcronym{cd}{
    short = CD ,
    long  = compact disc
  }
\end{sourcecode}
Then the usage is
\begin{example}[side-by-side]
  \begin{tabular}{ll}
    first  & \ac{cd} \\
    second & \ac{cd} \\
    long   & \acl{cd} \\
    short  & \acs{cd} \\
    full   & \acf{cd}
  \end{tabular}
\end{example}

\section{Setting options}\label{sec:setting-options}
\subsection{Load-time options}\label{sec:load-time-options}
\acro\ knows only a small set of load-time options which can be used as
argument to \cs*{usepackage}:
\begin{options}
  \keychoice{version}{2,3}\Default{3}
    The option allows you to use the last version prior to the update to
    version~3.  This may help if you don't have the time to fix issues after
    upgrading to the new version.
  \keybool{upgrade}\Default{true}
    When this option is used \acro\ tries to give as much helpful and
    meaningful warning or error messages when a deprecated or removed command,
    property, or option is used.  This is especially useful if you are
    upgrading from version~2.  The option will initially be true for a few
    months after the upgrade to version~3.
\end{options}

\subsection{Setup command}\label{sec:setup-command}
All options of \acro\ that have \emph{not} been mentioned in
section~\ref{sec:load-time-options} have to be set up either with this command
\begin{commands}
  \command{acsetup}[\marg{options}]
    or as option to other commands.  If the latter is possible then it is
    described when the corresponding commands are explained.  Options usually
    follow a key\slash value syntax and are always described in the following
    way:
\end{commands}
\begin{options}
  \opt*{option}
    An option without a value. Those options are very rare if there are any.
  \keyval*{option}{value}\Default{preset}
    An option where a value can be given.  The pre-set value is given to the
    right.
  \keychoice*{option}{\default{choiceA},choiceB,choiceC}\Default{choiceB}
    An option with a determined set of choices. The underlined value is
    chosen if the option is given without value.
  \keybool*{option}
    A boolean option with only the choices \code{true} and \code{false}.
  \opt*{option}\Module{module}
    An option at a deeper level belonging to the module \module*{module}.
\end{options}
All of the above is probably clear from an example (using real options):
\begin{sourcecode}
  \acsetup{
    make-links = true ,   % boolean
    index ,               % boolean
    format = \emph ,      % standard
    list / local ,        % boolean option of the list module
    list / display = all  % choice option of the list module
  }
\end{sourcecode}

\part{Comprehensive description of creation and usage of acronyms}\label{part:compr-descr-creat}
\section{Declaring acronyms and other abbreviations}\label{sec:decl-acronyms-other}

All acronyms have to be declared in the preamble with the following command in
order to be used in the document. Any usage of an acronym which has not been
declared leads to an error message.
\begin{commands}
  \command{DeclareAcronym}[\marg{id}\marg{list of properties}]
    The basic command for declaring an acronym where \meta{id} is a unique
    string identifying the acronym.  Per default this is case sensitive which
    means \code{id} is different from \code{ID}, for example.
\end{commands}
The command understands a number of properties which are listed in the
following sections.  This is a comprehensive overview over the existing
properties. Many properties are also explained in more detail in later
sections of this manual.
\begin{options}
  \keybool{case-sensitive}\Default{false}
    When this is set you can write the \ac{id} of the acronym upper- or lower-
    or mixed case and it is recognized by \acro\ as the same.  This might be
    useful when the acronym appears in the page header, for example.
\end{options}

\begin{bewareofthedog}
  In its simplest form an acronym needs a short and a long form.  Please note
  that both properties \emph{must} be set.
\end{bewareofthedog}

\subsection{Basic properties}\label{sec:basic-properties}
\begin{properties}
  %% short
  \propval{short}{text}\Default!
    The short form of the acronym.  \emph{This property is required}: an
    acronym must have a short form.
\end{properties}
Maybe you mostly have simple acronyms where the \ac{id} and short form are
the same.  In that case you can use
\begin{options}
  \keybool{use-id-as-short}\Default{false}
    to use the \ac{id} of the acronym as short form. For more complicated
    cases this would still allow you to set the short form.
\end{options}
\begin{properties}
  %% long
  \propval{long}{text}\Default!
    The long form of the acronym.  \emph{This property is required}: an
    acronym must have a description.
  %% alt
  \propval{alt}{text}\Default
    Alternative short form.
  %% extra
  \propval{extra}{text}\Default
    Extra information to be added in the list of acronyms.
  %% foreign
  \propval{foreign}{long form in foreign language}\Default
    Can be useful when dealing with acronyms in foreign languages, see
    section~\vref{sec:fore-lang-acronyms} for details.
  %% long-post
  \propval{long-post}{text}\Default
    \meta{text} is appended to the long form of the acronym in the text but
    not in the list of acronyms.
  %% post
  \propval{post}{text}\Default
    \meta{text} is appended to the acronym in the text but not in the list of
    acronyms.
  %% single
  \propval{single}{text}\Default={long}
    If provided \meta{text} will be used instead of the long form if the
    acronym is only used a single time \emph{and} the option
    \option{single} has been set, see section~\vref{sec:single-appe-an}.
  %% sort
  \propval{sort}{text}\Default={short}
    If used the acronym will be sorted according to this property instead of
    its \ac{id}.
  %% tag
  \propval{tag}{csv list}\Default
    The tag(s) of an acronym.
  %% cite
  \proplit-{cite}{\oarg{prenote}\oarg{postnote}\marg{citation keys}}\Default
    A citation that is printed to the acronym according to an option explained
    later.
  %% before-citation
  \propval{before-citation}{text}\Default
    \meta{text} is prepended to the citation of the acronym when and where
    the citation is printed.
  %% index
  \propval{index}{text}\Default
    This property allows to overwrite the automatic index entry with an
    arbitrary one.  See section~\vref{sec:indexing} for details.
  %% index-sort
  \propval{index-sort}{text}\Default={sort}
    If you use the option \option{index} every occurrence of an acronym is
    recorded to the index and sorted by its short form or (if set) by the
    value of the \property{sort} property.  This property allows to set an
    individual sorting option for the index.  See section~\vref{sec:indexing}
    for details.
\end{properties}

\subsection{Properties related to plural and indefinite forms}\label{sec:prop-relat-plur}
\begin{properties}
  %% short-plural
  \propval{short-plural}{text}\Default{s}
    The plural ending appended to the short form.
  %% short-plural-form
  \propval{short-plural-form}{text}\Default
    The plural short form of the acronym; replaces the short form when used
    instead of appending the plural ending.
  %% long-plural
  \propval{long-plural}{text}\Default{s}
    The plural ending appended to the long form.
  %% long-plural-form
  \propval{long-plural-form}{text}\Default
    Plural long form of the acronym; replaces the long form when used
    instead of appending the plural ending.
   %% alt-plural
  \propval{alt-plural}{text}\Default{s}
    The plural ending appended to the alternative form.
  %% alt-plural-form
  \propval{alt-plural-form}{text}\Default
    The plural alternative form of the acronym; replaces the alternative form
    when used instead of appending the plural ending.
  %% foreign-plural
  \propval{foreign-plural}{text}\Default{s}
    The plural ending appended to the foreign form.
  %% foreign-plural-form
  \propval{foreign-plural-form}{text}\Default
    Plural foreign form of the acronym; replaces the foreign form when used
    instead of appending the plural ending.
  %% short-indefinite
  \propval{short-indefinite}{text}\Default{a}
    Indefinite article for the short form.
  %% long-indefinite
  \propval{long-indefinite}{text}\Default{a}
    Indefinite article for the long form.
  %% alt-indefinite
  \propval{alt-indefinite}{text}\Default{a}
    Indefinite article for the alternative form.
\end{properties}

\subsection{Properties related to formatting}\label{sec:prop-relat-form}
\begin{properties}
  \propval{format}{code}\Default
    The format used for both short and long form of the acronym.
  %% short-format
  \propval{short-format}{code}\Default={format}
    The format used for the short form of the acronym.
  %% long-format
  \propval{long-format}{code}\Default={format}
    The format used for the long form of the acronym.
  %% first-long-format
  \propval{first-long-format}{code}\Default={long-format}
    The format used for the first appearance of the long form of the acronym.
  %% alt-format
  \propval{alt-format}{code}\Default={short-format}
    The format used for the alternative form of the acronym. If this is not
    given the short format will be used.
  %% extra-format
  \propval{extra-format}{code}\Default
    The format used for the additional information of the acronym.
  %% foreign-format
  \propval{foreign-format}{code}\Default
    The format used for the foreign form of the acronym.
  %% single-format
  % \propval{single-format}{code}\Default={long-format}
  %   The format used for the acronym if the acronym is only used a single
  %   time.
  %% list-format
  \propval{list-format}{code}\Default={long-format}
    The format used for the long form of the acronym in the list if the list
    template supports it. All pre-defined list templates \emph{do} support
    it.
  %% first-style
  \propchoice{first-style}{long-short,short-long,short,long,footnote}\Default
    The style of the first appearance of the acronym, see also
    section~\vref{sec:first-or-full}.
  %% first-style
  \propchoice{subsequent-style}{long-short,short-long,short,long,footnote}\Default
    \sinceversion{3.4}The style of the appearance of the acronym after the
    first time.
  %% single-style
  \propchoice{single-style}{long-short,short-long,short,long,footnote}\Default
    The style of a single appearance of the acronym, see also
    section~\vref{sec:single-appe-an}.
\end{properties}

\subsection{Properties related to the created \acs*{pdf} file}\label{sec:prop-relat-creat}
\begin{properties}
  %% pdfstring
  \propval{pdfstring}{pdfstring}\Default={short}
    Used as \acs{pdf} string replacement in bookmarks when used together with
    the \pkg{hyperref}~\cite{pkg:hyperref} or the \pkg{bookmark}
    package~\cite{pkg:bookmark}.
  %% pdfcomment
  \propval{pdfcomment}{text}
    Sets a tooltip description for an acronym.  For actually getting
    tooltips you also need an appropriate setting of the options
    \option{pdfcomment/cmd} and \option{pdfcomment/use}, see also
    section~\vref{sec:pdf-comments}.
  %% short-acc
  \propval{short-acc}{text}\Default={short}
    Sets the \code{ActualText} property as presented by the \pkg{accsupp}
    package for the short form of the acronym.
  %% long-acc
  \propval{long-acc}{text}\Default={long}
    Sets the \code{ActualText} property as presented by the \pkg{accsupp}
    package for the long form of the acronym.
  %% alt-acc
  \propval{alt-acc}{text}\Default={alt}
    Sets the \code{ActualText} property as presented by the \pkg{accsupp}
    package for the alternative short form of the acronym.
  %% foreign-acc
  \propval{foreign-acc}{text}\Default={foreign}
    Sets the \code{ActualText} property as presented by the \pkg{accsupp}
    package for the foreign form of the acronym.
  %% extra-acc
  \propval{extra-acc}{text}\Default={extra}
    Sets the \code{ActualText} property as presented by the \pkg{accsupp}
    package for the extra information of the acronym.
  %% single-acc
  \propval{single-acc}{text}\Default={long-acc}
    Sets the \code{ActualText} property as presented by the \pkg{accsupp}
    package for a single appearance of the acronym.
  %% list-acc
  \propval{list-acc}{text}\Default={list}
    Sets the \code{ActualText} property as presented by the \pkg{accsupp}
    package for the appearance in the list of acronyms.
\end{properties}

\subsection{Futher properties}\label{sec:futher-properties}
\begin{properties}
  %% list
  \propval{list}{text}\Default={long}
    If specified this will be written in the list as description instead of
    the long form if the corresponding list template supports it.
  %% foreign-babel
  \propval{foreign-babel}{language}\Default
    The \pkg{babel}~\cite{pkg:babel} or
    \pkg{polyglossia}~\cite{pkg:polyglossia} language of the foreign form.
    This language is used to wrap the entry with
    \cs*{foreignlanguage}\marg{language} if either \pkg{babel} or
    \pkg{polyglossia} is loaded.  You'll need to take care that the
    corresponding language is loaded by \pkg{babel} or \pkg{polyglossia}.
  %% foreign-locale
  \propval{foreign-locale}{language}\Default
    The language name that is output when the option
    \module{locale}\code{/}\option{display} is used.  If this property is not
    set then the appropriate value might be derived from
    \property{foreign-babel}. See section~\vref{sec:fore-lang-acronyms} for
    details.
\end{properties}

\section{Using acronyms}\label{sec:using-acronyms}
There are a number of commands to use acronyms with. Their names always follow
the same pattern which should make their usage intuitive immediately.

All of these commands have a starred form which means \enquote{don't count
  this as usage}. All of these commands also have an optional argument that
allows to set options for that usage only.
\begin{commands}
  \command*{acrocommand}[\sarg\oarg{options}\marg{id}]
    This is the general syntax of all of the commands listed below.  The star
    and the optional argument is left way for the sake of readability.
    \emph{A command \cs*{acrocommand} does not actually exist.}
  \command{ac}[\marg{id}\quad\cs{Ac}\marg{id}\quad\cs{acp}\marg{id}\quad
    \cs{Acp}\marg{id}\quad\cs{iac}\marg{id}\quad\cs{Iac}\marg{id}]
    \cs{ac} prints the acronym \meta{id}, the first time with full description
    and every subsequent use only the abbreviated form. \cs{Ac} does the same
    but uppercases the first letter -- this may be needed at the beginning of
    a sentence.  The commands \cs{acp} and \cs{Acp}, resp., print the
    corresponding plural forms.  The commands \cs{iac} and \cs{Iac}, resp.,
    print indefinite forms.
  \command{acs}[\marg{id}\quad\cs{Acs}\marg{id}\quad\cs{acsp}\marg{id}\quad
    \cs{Acsp}\marg{id}\quad\cs{iacs}\marg{id}\quad\cs{Iacs}\marg{id}]
    \cs{acs} prints the short form of the acronym \meta{id}. \cs{Acs} does the
    same but uppercases the first letter. The commands \cs{acsp} and
    \cs{Acsp}, resp., print the corresponding plural forms. The commands
    \cs{iacs} and \cs{Iacs}, resp., print indefinite forms.
  \command{acl}[\marg{id}\quad\cs{Acl}\marg{id}\quad\cs{aclp}\marg{id}\quad
    \cs{Aclp}\marg{id}\quad\cs{iacl}\marg{id}\quad\cs{Iacl}\marg{id}]
    \cs{acl} prints the long form of the acronym \meta{id}. \cs{Acl} does the
    same but uppercases the first letter. The commands \cs{aclp} and
    \cs{Aclp}, resp., print the corresponding plural forms. The commands
    \cs{iacl} and \cs{Iacl}, resp., print indefinite forms.
  \command{aca}[\marg{id}\quad\cs{Aca}\marg{id}\quad\cs{acap}\marg{id}\quad
    \cs{Acap}\marg{id}\quad\cs{iaca}\marg{id}\quad\cs{Iaca}\marg{id}]
    \cs{aca} prints the alternative short form of the acronym \meta{id}.
    \cs{Aca} does the same but uppercases the first letter. The commands
    \cs{acap} and \cs{Acap}, resp., print the corresponding plural forms. The
    commands \cs{iaca} and \cs{Iaca}, resp., print indefinite forms.
  \command{acf}[\marg{id}\quad\cs{Acf}\marg{id}\quad\cs{acfp}\marg{id}\quad
    \cs{Acfp}\marg{id}\quad\cs{iacf}\marg{id}\quad\cs{Iacf}\marg{id}]
    \cs{acf} prints the full form of the acronym \meta{id}. \cs{Acf} does the
    same but uppercases the first letter. The commands \cs{acfp} and
    \cs{Acfp}, resp., print the corresponding plural forms. The commands
    \cs{iacf} and \cs{Iacf}, resp., print indefinite forms.
\end{commands}

The usage should be clear. Let's assume you have defined an acronym
\acs*{ufo}\label{ufo} like this:
\begin{sourcecode}
  \DeclareAcronym{ufo}{
    short = UFO ,
    long = unidentified flying object ,
    foreign = unbekanntes Flugobjekt ,
    foreign-plural-form = unbekannte Flugobjekte ,
    foreign-babel = ngerman ,
    long-indefinite = an
  }
\end{sourcecode}
The typical outputs look like this:
\begin{example}
  \ac{ufo} \\
  \iac{ufo} \\
  \iacl{ufo} \\
  \Iacf{ufo} \\
  \acfp{ufo}
\end{example}

\begin{bewareofthedog}
  In a number of contexts all acronym commands act as if their starred form is
  used: in the table of contents, in the list of figures, and in the list of
  tables.  The same is true for floats and the measuring phase of common
  table environments like \env*{tabularx} or \env*{ltxtable}.
\end{bewareofthedog}

\section{Alternative short forms}\label{sec:altern-short-forms}
Sometimes expressions have two different short forms. And example might be
\acs*{jpg} which also often is \aca*{jpg}. This is what the property
\property{alt} is there for.
\begin{properties}
  \propval{alt}{text}
    Alternative short form.
\end{properties}
Let's define \acs*{jpg}:
\begin{sourcecode}
  \DeclareAcronym{jpg}{
    short = JPEG ,
    sort  = jpeg ,
    alt   = JPG ,
    long  = Joint Photographic Experts Group
  }
\end{sourcecode}
And let's see how to use it:
\begin{example}
  \ac{jpg} \\
  \ac{jpg} \\
  \aca{jpg}
\end{example}
As you can see the full form shows both short forms of the acronym. This could
be changed by altering the template for the full form, see
section~\vref{sec:templates} and section~\vref{sec:first-or-full}.  The
alternative form is also printed in the list of acronyms, see
section~\vref{sec:acronyms}.  This can also be changed by altering the
template for the list, again see section~\ref{sec:templates}.

\section{The first or full appearance}\label{sec:first-or-full}
If an acronym is used for the first time with \cs{ac} (after any number of
usages with the starred forms of the usage commands listed in
section~\vref{sec:using-acronyms}) or if an acronym is used \cs{acf}, then the
first or full appearance of the acronym is printed\footnote{This usually
  requires at least two \LaTeX\ runs until it is stable.}.

The first or full appearance of an acronym is determined by this option:
\begin{options}
  \keychoice{first-style}{long-short,short-long,short,long,footnote}\Default{long-short}
    The style of the first appearance of the acronym. This options sets the
    appearance for all acronyms.  Available options in reality are the names
    of all defined templates of the type \code{acronym}. All pre-defined
    templates can be found in section~\vref{sec:pre-defin-templ}.
  \keychoice{subsequent-style}{long-short,short-long,short,long,footnote}\Default{short}
    \sinceversion{3.4}The style of the appearance of the acronym after the
    first time. This options sets the appearance for all acronyms.  Available
    options in reality are the names of all defined templates of the type
    \code{acronym}. All pre-defined templates can be found in
    section~\vref{sec:pre-defin-templ}.
\end{options}
It might be desirable to set the first appearance of an acronym
individually. This is possible by setting the corresponding property:
\begin{properties}
  \propchoice{first-style}{long-short,short-long,short,long,footnote}\Default
    The style of the first appearance of the acronym.
\end{properties}
Let's again look at an example:
\begin{example}[side-by-side]
  \acf[first-style=long-short]{cd} \\
  \acf[first-style=short-long]{cd} \\
  \acf[first-style=footnote]{cd} \\
  \acf[first-style=long]{cd} \\
  \acf[first-style=short]{cd}
\end{example}
This also demonstrates the use of the optional argument.

An example of an abbreviation that should have \code{long} as first appearance
might be \enquote{\acs*{etc}}, defined like this
\begin{sourcecode}
  \DeclareAcronym{etc}{
    short = etc\acdot ,
    long = et cetera ,
    format = \textit ,
    first-style = long ,
    plural = 
  }
\end{sourcecode}
and output like this:
\begin{example}[side-by-side]
  \ac{etc}, \ac{etc} \ac{etc}.
\end{example}
The command \cs{acdot} is explained in section~\vref{sec:trailing-tokens}.
Basically it checks if a dot follows and outputs a dot if not.

\section{Single appearances of an acronym}\label{sec:single-appe-an}
If an acronym is used only once (not counting usages with the starred forms of
the usage commands listed in section~\vref{sec:using-acronyms}), then the
single appearance of the acronym is printed\footnote{This usually requires at
  least two \LaTeX\ runs until it is stable.}.

The single appearance of an acronym is determined by this option:
\begin{options}
  \keychoice{single}{\default{true},false,\meta{number}}\Default{false}
    This option determines wether a single appearance of an acronym counts as
    \emph{usage}. It might be desirable in such cases that an acronym is
    simply printed as long form and not added to the list of acronym.  This is
    what this option does.  With \meta{number} the minimal number of usages
    can be given that needs to be exceeded.  \keyis{single}{1} is the same as
    \keyis{single}{true}.
  \keychoice{single-style}{long-short,short-long,short,long,footnote}\Default{long}
    The style of the single appearance of an acronym.  Can be used to
    determine how a single appearance is printed if the option \option{single}
    has been set. This option sets the appearance for all acronyms.
    Available options in reality are the names of all defined templates of the
    type \code{acronym}. All pre-defined templates can be found in
    section~\vref{sec:pre-defin-templ}.
\end{options}

If you like you can also set the single appearance of an acronym individually:
\begin{properties}
  \propval{single}{text}\Default
    If provided \meta{text} will be used instead of whatever template
    (\enquote{style}) has been set for the single appearance if the acronym is
    only used a single time \emph{and} the option \option{single} has been
    set\footnote{Actually the template \template{single} is used which
      typesets the \property{single} property.}.
  % \propval{single-format}{code}\Default={long-format}
  %   The format used for the acronym if the acronym is only used a single
  %   time.
  \propchoice{single-style}{long-short,short-long,short,long,footnote}\Default
    The style of the single appearance of the acronym.
\end{properties}
Let's again look at an example. The acronym \acs*{png} is defined as follows:
\begin{sourcecode}
  \DeclareAcronym{png}{
    short = PNG ,
    long  = Portable Network Graphics ,
    first-style = short-long ,
    single-style = short
  }
\end{sourcecode}
And it is used only once in this manual\footnote{You will find it in the list
  of acronyms in section~\ref{sec:acronyms} nonetheless as this document does
  \keyis{list/display}{all}.}:
\begin{example}[side-by-side]
  \ac{png}
\end{example}  
Please be aware that \cs{acf} would still print the full form, of course.

\section{Printing the list}\label{sec:printing-list}
\subsection{The main command and its options}\label{sec:main-list}
The main idea is simple: just place
\begin{commands}
  \command{printacronyms}[\oarg{options}]
    where you want the list to appear.  It may require several (most times
    two) \LaTeX\ runs for it to stabilize so look out for any warnings
    from \acro\ requiring to re-run.
\end{commands}
The options controlling the list are these:
\begin{options}
  \keychoice{template}{description,table,longtable,lof,toc}%
    \Module{list}\Default{description}
    Choose the template to create the list with.  See more on this in
    sections~\vref{sec:templates} and~\vref{sec:examples}.
  \keybool{sort}\Module{list}\Default{true}
    Decide wether to sort the list of acronyms alphabetically or to print it
    in order of definition.
  \keychoice{display}{all,used}\Module{list}\Default{used}
    Decide wether to print only the acronyms actually used in the document or
    all acronyms which have been declared in the preamble.
  \keyval{exclude}{csv list of tags}\Module{list}\Default
    Set a list of tags to exclude from the list.  Only acronyms not
    belonging to one of these tags will be included.
  \keyval{include}{csv list of tags}\Module{list}\Default
    Set a list of tags to include in the list.  Only acronyms belonging to
    one of these tags will be included.
  \keybool{add}\Module{list}\Default
    \sinceversion{3.4}Set a list of tags to include in the list.  These
    acronyms will be included in any case.
  \keychoice{heading}{none,section,section*,chapter,chapter*}\Module{list}
    Choose the heading template for the list of acronyms.

    This only has an effect if the list template supports it.  All pre-defined
    templates \emph{do} support it.
  \keyval{name}{text}\Module{list}\Default{\cs{acrotranslate}\Marg{list-name}}
    Overwrites the text which is used in the heading.

    This only has an effect if the list template supports heading templates
    \emph{and} the heading templates support it.  All pre-defined heading
    templates \emph{do} support this.
  \keyval{preamble}{text}\Module{list}\Default
    Set a preamble to be placed between heading and actual list.

    This only has an effect if the list template supports it.  All pre-defined
    templates \emph{do} support it.
  \keybool{display}\Module{list,locale}\Default{false}
    This options determines wether the language of the foreign form is printed
    or not.

    This only has an effect if the list template supports foreign forms.  All
    pre-defined templates \emph{do} support them.
\end{options}

All these options can be set with \cs{acsetup} globally or locally as options
to \cs{printacronyms}. In the latter case omit the leading
\module{list}:
\begin{sourcecode}
  \acsetup{list/display=all,list/exclude=units}
  or
  \printacronyms[display=all,exclude=units]
\end{sourcecode}

\subsection{Add page numbers to the list}\label{sec:page-numbers}
If you want to include the page numbers where the acronyms have been used in
the list of acronym you can use these options:
\begin{options}
  \keychoice{display}{first,all,none}\Module{pages}\Default{none}
    Decide wether to include page numbers in the list of acroynms and wether
    to add the first page or every page.  When you choose \code{first} and
    have \pkg{hyperref} loaded you will also get a backlink to that page.
  \keybool{use}\Module{pages,seq}\Default{true}
    Turns a two-page range into \meta{num}\,f.\ (\latin{sequens}) and a
    three-page range into \meta{num}\,ff.\ (\latin{sequentes}) when set to
    \code{true}.
  \keyval{pre}{code}\Module{pages,seq}\Default{\cs*{,}}
    \meta{code} is inserted between the page number and the sequens or
    sequentes symbol.
  \keyval{threshold}{num}\Module{pages,seq}\Default{3}
    The threshold for a page range to be turned into \latin{sequentes}.  A
    page range above the threshold is still typeset as a range:
    \meta{num1}--\meta{num2}.
  \keyval{fill}{code}\Module{pages}\Default{\cs*{dotfill}}
    This is the code that is placed between acronym description and actual
    page numbers.
  \keybool{name}\Module{pages}\Default{false}
    If set to true the page numbers are preceded with p.\ or pp.
\end{options}
Additionally\sinceversion{3.3} to setting these options with \cs{acsetup} they
can be set as options to \cs{printacronyms}:
\begin{sourcecode}
  \printacronyms[pages={display=all,seq/use=false}]
\end{sourcecode}

\subsection{Filter lists using tags}\label{sec:lists-tags}
With the property \property{tag} you can assign one or more tags to an
acronym.  These tags can be used to filter the list of acronyms.
\begin{properties}
  %% tag
  \propval{tag}{csv list}\Default
    The tag(s) of an acronym.
\end{properties}
\begin{options}
  \keyval{exclude}{csv list of tags}\Module{list}\Default
    Set a list of tags to exclude from the list.  Only acronyms not
    belonging to one of these tags will be included.
  \keyval{include}{csv list of tags}\Module{list}\Default
    Set a list of tags to include in  the list.  Only acronyms belonging to
    one of these tags will be included.
\end{options}
  
Let's look at an example. This manual declares these two acronyms with the tag
\code{city}:
\begin{sourcecode}
  \DeclareAcronym{la}{
    short = LA ,
    long = Los Angeles,
    plural = ,
    tag = city
  }
  \DeclareAcronym{ny}{
    short = NY ,
    long = New York ,
    plural = ,
    tag = city
  }
\end{sourcecode}
We can now use this to either print a list \emph{without} these acronyms by
saying
\begin{sourcecode}
  \printacronyms[exclude=city]
\end{sourcecode}
or print a list \emph{with only} these acronyms with
\begin{example}
  \printacronyms[include=city,heading=none]
\end{example}
\begin{bewareofthedog}
  If you use both \option{exclude} and \option{include} and list a tag in both
  \option{exclude} takes precedence over \option{include}.
  \begin{sourcecode}
  \printacronyms[exclude={a,b},include={b,c}]
  \end{sourcecode}
  would only print acronyms with tag \code{c}.
\end{bewareofthedog}

\subsection{Local lists}\label{sec:local-lists}
Maybe you like a list of acronyms for each chapter in a book which only lists
the acronyms used within this chapter.  You need to do three things: set
\begin{options}
  \keybool{use}\Module{barriers}\Default{false}
    this option to \code{true}, place
\end{options}
\begin{commands}
  \command{acbarrier}
    before a new chapter starts (this is not necessary for the first chapter),
    and use \cs{printacronyms} with the option
\end{commands}
\begin{options}
  \keybool{local}\Module{list}\Default{false}
    or set this option once in the preamble with \cs{acsetup} so it is applied
    to every list.
\end{options}
Please read more on barriers in section~\vref{sec:barriers}.
\begin{bewareofthedog}
  Please don't use page numbers together with local lists for the time being.
  If an acronym appears in more than one list both lists would contain the
  \emph{same} page numbers anstead of only the ones local to barriers.

  For the similar reasons please also don't use \option{make-links} together
  with local lists.

  This \emph{might} be resolved on day.
\end{bewareofthedog}

\section{Formatting}\label{sec:formatting}
\acro\ has a number of options and parameters which can be used to influence
the formatting of acronyms.
\begin{options}
  \keyval{format}{code}\Default
    Sets the format for both the short and the long form.
  \keyval{short}{code}\Module{format}\Default
    Sets the format for the short form.
  \keyval{long}{code}\Module{format}\Default
    Sets the format for the long form.
  \keyval{first-long}{code}\Module{format}\Default
    Sets the format for the first appearance of the long form.
  \keyval{alt}{code}\Module{format}\Default
    Sets the format for the alternative form.
  \keyval{extra}{code}\Module{format}\Default
    Sets the format for the extra information.
  \keyval{foreign}{code}\Module{format}\Default
    Sets the format for the foreign form.
  \keyval{list}{code}\Module{format}\Default
    Sets the format for the long form in the list form.
\end{options}
While this options influence the formatting of the acronyms globally you can
also give each acronym its own formatting individually:
\begin{properties}
  \propval{format}{code}\Default
    The format used for both short and long form of the acronym.
  %% short-format
  \propval{short-format}{code}\Default={format}
    The format used for the short form of the acronym.
  %% long-format
  \propval{long-format}{code}\Default={format}
    The format used for the long form of the acronym.
  %% first-long-format
  \propval{first-long-format}{code}\Default={long-format}
    The format used for the first appearance of the long form of the acronym.
  %% alt-format
  \propval{alt-format}{code}\Default={short-format}
    The format used for the alternative form of the acronym. If this is not
    given the short format will be used.
  %% extra-format
  \propval{extra-format}{code}\Default
    The format used for the additional information of the acronym.
  %% foreign-format
  \propval{foreign-format}{code}\Default
    The format used for the foreign form of the acronym.
  %% single-format
  \propval{single-format}{code}\Default={long-format}
    The format used for the acronym if the acronym is only used a single
    time.
  %% list-format
  \propval{list-format}{code}\Default={long-format}
    The format used for the long form of the acronym in the list if the list
    template supports it. All pre-defined list templates \emph{do} support
    it.
  %% first-style
  \propchoice{first-style}{long-short,short-long,short,long,footnote}\Default
    The style of the first appearance of the acronym, see also
    section~\vref{sec:first-or-full}.
  %% single-style
  \propchoice{single-style}{long-short,short-long,short,long,footnote}\Default
    The style of a single appearance of the acronym, see also
    section~\vref{sec:single-appe-an}.
\end{properties}
Per\changedversion{3.3} default the individual formatting instructions
\emph{replace} the global ones. This can be changed through the option
\begin{options}
  \keybool{replace}\Module{format}\Default{true}
    With this option active local options will \emph{replace} the global
    ones.
\end{options}

Let's see an example:
\begin{sourcecode}
  \DeclareAcronym{pdf}{
    short = pdf ,
    long = Portable Document Format ,
    short-format = \scshape
  }
\end{sourcecode}
\begin{example}
  \acsetup{format = \itshape}
  \acf{pdf} \par
  \acsetup{format/replace=false}
  \acf{pdf}
\end{example}

\section{Plural forms and other endings}\label{sec:plural-forms-other}
\subsection{The plural ending and the plural form}\label{sec:plural-ending-form}
Not in all languages plural forms are as easy as always appending an
\enquote{s}.  Not even English.  Sometimes there's other endings
instead\footnote{German is full of such examples.}.  This is why \acro\ has
quite a number of different properties related to plural forms or endings:
\begin{properties}
  %% short-plural
  \propval{short-plural}{text}\Default{s}
    The plural ending appended to the short form.
  %% short-plural-form
  \propval{short-plural-form}{text}\Default
    The plural short form of the acronym; replaces the short form when used
    instead of appending the plural ending.
  %% long-plural
  \propval{long-plural}{text}\Default{s}
    The plural ending appended to the long form.
  %% long-plural-form
  \propval{long-plural-form}{text}\Default
    Plural long form of the acronym; replaces the long form when used
    instead of appending the plural ending.
   %% alt-plural
  \propval{alt-plural}{text}\Default{s}
    The plural ending appended to the alternative form.
  %% alt-plural-form
  \propval{alt-plural-form}{text}\Default
    The plural alternative form of the acronym; replaces the alternative form
    when used instead of appending the plural ending.
  %% foreign-plural
  \propval{foreign-plural}{text}\Default{s}
    The plural ending appended to the foreign form.
  %% foreign-plural-form
  \propval{foreign-plural-form}{text}\Default
    Plural foreign form of the acronym; replaces the foreign form when used
    instead of appending the plural ending.  
\end{properties}
There are two options which allow to change the default values for the whole
document:
\begin{options}
  \keyval{short-plural-ending}{text}\Default{s}
    Defines the plural ending for the short forms to be \meta{text}.
  \keyval{long-plural-ending}{text}\Default{s}
    Defines the plural ending for the long forms to be \meta{text}.
\end{options}
Now let's see two simple examples demonstrating the two different kinds of plural
settings:
\begin{sourcecode}
  \DeclareAcronym{sw}{
    short = SW ,
    long = Sammelwerk ,
    long-plural = e
  }
  \DeclareAcronym{MP}{
    short = MP ,
    long  = Member of Parliament ,
    plural-form = Members of Parliament
  }
\end{sourcecode}
The first one has another plural ending than the usual \enquote{s}. The second
one has a different plural form altogether because appending an \enquote{s}
would give a wrong form:
\begin{example}[side-by-side]
  \acfp{sw} \par
  \acfp{MP}
\end{example}

\subsection{Other endings}\label{sec:other-endings}
Besides plural endings there are other ones like the genitive case, for
example.  This is why \acro\ generalized the concept.
Section~\vref{sec:endings} explains in detail how to define and use additional
endings.

\section{Articles}
\subsection{Indefinite forms}\label{sec:indefinite-forms}
Indefinite forms can be a problem if the short and the long form of acronyms
have different indefinite articles\footnote{This may very well be a language
  specific issue.}.
\begin{example}[side-by-side]
  \acreset{ufo}%
  a \ac{ufo} \par
  an \ac{ufo}
\end{example}
And what good would it be to use a package like \acro\ if you have to keep
track of first and second uses, anyway?  This is why \acs{ufo} should be defined
like we did on page~\pageref{ufo}.  We then can just use the dedicated
commands and let them decide for us:
\begin{example}[side-by-side]
  \acreset{ufo}%
  \iac{ufo} \par
  \iac{ufo}
\end{example}
The commands which also output the indefinite article all start with an
\enquote{i} and have all been described in section~\vref{sec:using-acronyms}
already: \cs{iac}, \cs{Iac}, \cs{iacs}, \cs{Iacs}, \cs{iacl}, \cs{Iacl},
\cs{iaca}, \cs{Iaca}, \cs{iacf}, and \cs{Iacf}.

\subsection{Other articles}\label{sec:other-articles}
There might be cases -- most likely depending on your language -- when you
would like to have other articles behaving similar to the indefinite ones.
Section~\ref{sec:articles} explains in detail how to define and use additional
articles.

\section{Foreign language acronyms}\label{sec:fore-lang-acronyms}
Sometimes and in some fields more often than in others abbreviations are used
that are derived from another language.  \acro\ provides a number of
properties for such cases:
\begin{properties}
  %% foreign
  \propval{foreign}{long form in foreign language}\Default
    Can be useful when dealing with acronyms in foreign languages, see
    section~\vref{sec:fore-lang-acronyms} for details.
  %% foreign-plural
  \propval{foreign-plural}{text}\Default{s}
    The plural ending appended to the foreign form.
  %% foreign-plural-form
  \propval{foreign-plural-form}{text}\Default
    Plural foreign form of the acronym; replaces the foreign form when used
    instead of appending the plural ending.
  %% foreign-format
  \propval{foreign-format}{code}\Default
    The format used for the foreign form of the acronym.
  %% foreign-babel
  \propval{foreign-babel}{language}\Default
    The \pkg{babel} or \pkg{polyglossia} language of the foreign form. This
    language is used to wrap the entry with
    \cs*{foreignlanguage}\marg{language} if either \pkg{babel} or
    \pkg{polyglossia} is loaded.  You'll need to take care that the
    corresponding language is loaded by \pkg{babel} or \pkg{polyglossia}.
  %% foreign-locale
  \propval{foreign-locale}{language}\Default
    The language name that is output when the option
    \module{locale}\code{/}\option{display} is used.  If this property is not
    set then the appropriate value might be derived from
    \property{foreign-babel}.
\end{properties}
There are also some options:
\begin{options}
  \keybool{display}\Module{foreign}\Default{true}
    Determine wether to hide or display the foreign form.
  \keybool{display}\Module{list,foreign}\Default{true}
    \sinceversion{3.2}Determine wether to hide or display the foreign form in
    the list of acronyms.
  \keybool{display}\Module{locale}\Default{false}
    This options determines wether the language of the foreign form is printed
    or not when the full form of the acronym is printed.
  \keybool{display}\Module{list,locale}\Default{false}
    The same but for the list of acronyms.
  \keyval{format}{code}\Module{locale}\Default{\cs*{em}\cs*{text\_titlecase\_first:n}}
    Determines how said language is formatted when printed.  The last command
    in \meta{code} may take a mandatory argument.
\end{options}

Let's say you are writing a German document and are using the abbreviation
\acs*{ecu} for \acl*{ecu} which stems from the English \enquote{Electronic
  Control Unit}.  Then you can define it as follows:
\begin{sourcecode}
  \DeclareAcronym{ecu}{
    short   = ECU ,
    long    = Steuergerät ,
    foreign = Electronic Control Unit ,
    foreign-babel = english ,
    foreign-locale = englisch
  }
\end{sourcecode}
Now the abbreviation is introduced so that everyone understands the confusion:
\begin{example}
  \ac{ecu} \par
  \acsetup{locale/display,locale/format=\emph}
  \acf{ecu}
\end{example}

The property \property{foreign-babel} is used for ensuring correct hyphenation
as long as you use \pkg{babel} or \pkg{polyglossia} and load the corresponding
language, too.  If you are writing your document in English then \acro\ is
able to deduce the language used for the \enquote{locale} field by itself:
\begin{sourcecode}
  \DeclareAcronym{eg}{
    short = e.g\acdot ,
    long  = for example ,
    foreign = exempli gratia ,
    foreign-babel  = latin ,
    short-format = \textit ,
    foreign-format = \textit
  }
\end{sourcecode}
\begin{example}
  \acsetup{locale/display,first-style=short-long}
  \acf{eg}
\end{example}

\section{Uppercasing}\label{sec:uppercasing}

Depending on the kind of abbreviations you have and depending on their
definition and maybe also depending on your language the long and sometimes
also the short forms need to start with an uppercase letter at the beginning
of a sentence while it starts with a lowercase letter otherwise.

For this \acro\ provides uppercase versions for all predefined acronym
commands listed in section~\ref{sec:using-acronyms}.  The usage is
self-explaining:
\begin{example}
  There was \iacl{ufo} hovering \dots \par
  \Aclp{ufo} were hovering \dots
\end{example}
If you defined them with uppercase letters to begin with then these commands
have no effect, of course.
\begin{sourcecode}
  \DeclareAcronym{ufo}{
    short = UFO ,
    long = Unidentified Flying Object
  }
\end{sourcecode}

\pagebreak

There are a number of options to control the uppercasing behavior:
\begin{options}
  \opt{first}\Module{uppercase}
    The default setting. Converts the first letter to uppercase.
  \opt{title}\Module{uppercase}
    This is just a synonym of \option{first}.
  \opt{all}\Module{uppercase}
    Converts \emph{all} letters to uppercase.
  \opt{none}\Module{uppercase}
    Converts \emph{all} letters to \emph{lowercase}
  \keyval{cmd}{command}\Module{uppercase}
    All of the above options just choose the right command using this option
    internally. This means you can choose a different behavior altogether by
    setting this option to something else.  For example you could use
    \cs*{capitalisewords} from the package
    \pkg{mfirstuc}~\cite{pkg:mfirstuc}. The command needs to have one
    mandatory argument.
\end{options}

There may be reasons to exclude short forms from being uppercased.  This can
be controlled by this option:
\begin{options}
  \keybool{short}\Module{uppercase}\Default{true}
    It allows you to disable the mechanism for the \property{short} and
    \property{alt} properties.
\end{options}

\section{Citing and indexing}\label{sec:citing-indexing}
\subsection{Citing}\label{sec:citing}
Acronyms can be given cite keys.  This makes it possible to add a citation
reference automatically when the acronym is used for the first time.

Let's see an example first. \acs*{ny} has been defined like this:
\begin{sourcecode}
  \DeclareAcronym{ny}{
    short = NY ,
    long = New York ,
    plural = ,
    tag = city ,
    cite = NewYork
  }
\end{sourcecode}
The property \property{cite} will now trigger \acro\ to input
\verbcode+\cite{NewYork}+ after the acronym:
\begin{example}[side-by-side]
  \ac{ny}
\end{example}
Depending on the citation style (and probably other factors, too) it might be
desirable to add the citation rather inside the parentheses together with the
short form of the acronym and even cited with a different command.  For cases
like these \acro\ offers a number of options:\acuse{ny}
\begin{options}
  \keyval{cmd}{citation command}\Module{cite}\Default{\cs*{cite}}
    Choose the command with which citations ar printed.
  \keybool{group}\Module{cite}\Default{false}
    Decide wether to group citations with the short form in the parentheses.
    The template must support this.  \acro's pre-defined templates \emph{do}
    support it.
  \keychoice{display}{first,all,none}\Module{cite}\Default{first}
    Decide wether to output the citation in the first/full usage only or
    always or never.
  \keyval{pre}{text}\Module{cite}\Default{\cs*{nobreakspace}}
    Arbitrary code directly output before the citation.
  \keyval{cmd}{citation command}\Module{cite,group}\Default{\cs*{cite}}
    Choose the command with which grouped citations are printed.
  \keyval{pre}{text}\Module{cite,group}\Default{,\textvisiblespace}
    Arbitrary code directly output before the citation in the grouped case.
\end{options}

If for example you use \pkg{biblatex}'s \code{authoryear}
style~\cite{pkg:biblatex} you might want to have settings like these:
\begin{sourcecode}
  \acsetup{
    cite/group = true ,
    cite/cmd = \parencite ,
    cite/group/cmd = \cite
  }
\end{sourcecode}
\begin{cnltxcode}
\begin{lstlisting}[style=cnltx]
  \acsetup{cite/display = all}
  \acf{ny} \\
  \ac{ny}
\end{lstlisting}
\tcblower
  New York (NY, Wikipedia 2020) \\
  NY (Wikipedia 2020)
\end{cnltxcode}

\subsection{Indexing}\label{sec:indexing}
Maybe you want to add your acronyms to an index. In that case it is probably
desirable to let \acro\ make this automatically.  In the simplest case just
enable it:
\begin{options}
  \keybool{use}\Module{index}\Default{false}
    Enable indexing.
  \keyval{cmd}{index command}\Module{index}\Default{\cs*{index}}
    Choose a command for indexing.
  \keyval{disable}{code}\Module{index}\Default{\cs*{def}\cs*{@}\{\}}
    Sometimes it is desirable to change the meaning of a command inside an
    index entry.  For the entries created by \acro\ this can be achieved with
    this option.
  \opt{clear}\Module{index}
    This option clears the disable list.
\end{options}
While these options set global behavior there are also properties to set them
for an acronym individually.
\begin{properties}
  %% index
  \propval{index}{text}\Default
    This property allows to overwrite the automatic index entry with an
    arbitrary one.
  %% index-sort
  \propval{index-sort}{text}\Default={sort}
    If you use the option \option{index} every occurrence of an acronym is
    recorded to the index and sorted by its short form or (if set) by the
    value of the \property{sort} property.  This property allows to set an
    individual sorting option for the index.
  %% no-index
  \propbool{no-index}\Default{true}
    This property allows to exclude an acronym from being indexed.
\end{properties}
This manual is an example for the indexing feature.  Each acronym from
section~\vref{sec:acronyms} that has been used in this manual is also listed
in the index.

%% TODO: document the label property

\section{Barriers}\label{sec:barriers}
The main purpose of the concept of barriers is to be able to have \emph{local}
lists of acronyms.  This concept does a little bit more than that, though,
which should become clear from the following options:
\begin{options}
  \keybool{use}\Module{barriers}\Default{false}
    Activate usage of barriers.  Otherwise the command \cs{acbarrier} just
    does nothing except writing a warning in the log.
  \keybool{reset}\Module{barriers}\Default{false}
    When set to \code{true} the acronym usage is reset for all acronyms at a
    barrier.  The first use of \cs{ac} after a barrier will again look like
    the \cs{acf}.
  \keybool{single}\Module{barriers}\Default{false}
    When set to \code{true} a single usage of an acronym between two barriers
    with \cs{ac} will look according to the chosen style as explained in
    section~\vref{sec:single-appe-an}.  This option only has an effect when
    the option \option{single} is used as well.
\end{options}

There are two natural barriers in a document: \beginenv*\Marg{document} and
\endenv*\Marg{document}. You can add an arbitrary number of additional
barriers with
\begin{commands}
  \command{acbarrier}
    For this command to have any effect you must set
    \module{barriers}\code{/}\option{use} to \code{true}!
\end{commands}
\begin{bewareofthedog}
  It usually takes two or even three \LaTeX\ runs until acronym usages
  between barriers are properly counted.
\end{bewareofthedog}

\section{Trailing tokens}\label{sec:trailing-tokens}
\subsection{What is it about?}
\acro\ has the possibility to look ahead for certain tokens and switch a
boolean variable if it finds them.  Per default \acro\ knows about three
tokens: the \enquote{\code{dot}} (\sym{.}), the \enquote{\code{dash}}
(\sym{-}) and the \enquote{\code{babel-hyphen}} (\cs*{babelhyphen}).

You have seen an example for this already:
\begin{sourcecode}
  \DeclareAcronym{etc}{
    short = etc\acdot ,
    long = et cetera ,
    format = \textit ,
    first-style = long ,
    plural = 
  }
\end{sourcecode}
The macro \cs{acdot} recognizes if a dot is directly following. It only prints
a dot if it doesn't.
\begin{example}[side-by-side]
  \ac{etc} and \ac{etc}.
\end{example}

Another example: let's say you're a German scientist, you have
\begin{sourcecode}
  \DeclareAcronym{PU}{
    short = PU ,
    long = Polyurethan ,
    long-plural = e
  }
\end{sourcecode}
and you use it the first time like this:
\begin{sourcecode}
  \ac{PU}-Hartschaum
\end{sourcecode}
then according to German orthography and typesetting rules this should be
printed as
\begin{center}
  \enquote{Polyurethan(PU)-Hartschaum}
\end{center}
\ie, with \emph{no} space between long and short form.
\begin{example}[side-by-side]
  \acf{PU}-Hartschaum
\end{example}

This works because the template \code{long-short}\footnote{The template that
  is used by default for the first appearance.} uses \cs{acspace} at the
appropriate place and the manual setup does
\begin{sourcecode}
  \acsetup{trailing/activate = dash}
\end{sourcecode}
\cs{acspace} looks ahead for a trailing dash and adds a space it it doesn't
find it.

\subsection{How does it work?}
Tokens to look for can be defined and activated through the following options:
\begin{options}
  \keylit-{define}{\meta{token}\marg{name}}\Module{trailing}
    Defines token \meta{name} and tells \acro\ look for \meta{token} if
    \meta{name} is activated.
  \keyval{activate}{csv list of token names}\Module{trailing}
    Tell \acro\ to look for trailing tokens.  This is done by giving a csv
    list of the internal \emph{names} of the tokens.  Per default only
    \code{dot} is activated.
  \keyval{deactivate}{csv list of token names}\Module{trailing}
    Tell \acro\ not to look for trailing tokens.  This is done by giving a csv
    list of the internal \emph{names} of the tokens.
\end{options}
The package itself does this:
\begin{sourcecode}
  \acsetup{
    trailing/define   = . {dot} ,
    trailing/define   = {, {comma}} ,
    trailing/define   = - {dash}
    trailing/define   = \babelhyphen {babel-hyphen} ,
    trailing/activate = {dot,comma}
  }
\end{sourcecode}

In order to make use of this mechanism there is the following command:
\begin{commands}
  \expandable\command{aciftrailing}[\marg{csv list of token
    names}\marg{true}\marg{false}]
    Check if one of the tokens listed in \meta{csv list of token names} is
    following and either place \meta{true} or \meta{false} in the input
    stream.
\end{commands}
This command is used to define the two commands you already know:
\begin{commands}
  \expandable\command{acdot} Inserts \cs{abbrdot} if no \code{dot} follows.
  \expandable\command{acspace} Inserts a \cs*{space} if no \code{dash} or
    \code{babel-hyphen} follows.
  \expandable\command{abbrdot} Inserts \sym{.}\cs*{@}
\end{commands}
The definitions are equivalent to the following code:
\begin{sourcecode}
  \newcommand*\acdot{\aciftrailing{dot}{}{\abbrdot}}
  \newcommand*\acspace{\aciftrailing{dash,babel-hyphen}{}{\space}}
\end{sourcecode}
You are of course free to redefine them according to your needs.

\section{Using or resetting acronyms}\label{sec:using-or-resetting}
Sometimes it is necessary to mark an acronym as used before it actually has
been used or to mark an acronym as unused even though it \emph{has} been
used.  You have already seen one of the commands which make it possible:
\begin{commands}
  \command{acuse}[\marg{csv list of acronym ids}]
    Every acronym given in the list will be marked as used.
  \command{acuseall}
    Every acronym is marked as used.
  \command{acreset}[\marg{csv list of acronym ids}]
    Every acronym given in the list will be reset.
  \command{acresetall}
    Every acronym will be reset.
\end{commands}

\begin{bewareofthedog}
  In a number of contexts all acronym commands act as if their starred form is
  used: in the table of contents, in the list of figures, and in the list of
  tables.  The same is true for floats and the measuring phase of common
  table environments like \env*{tabularx} or \env*{ltxtable}.
\end{bewareofthedog}

\section{Bookmarks, backlinks and accessibility support}\label{sec:pdf-properties}
\subsection{Backlinks}\label{sec:backlinks}
When \acro\ is used together with the package
\pkg{hyperref}~\cite{pkg:hyperref} then you can make use of the following
option:
\begin{options}
  \keybool{make-links}\Default{false}
    If this is activated then every short or alternative appearance of an
    acronym will be linked to its description in the list of acronyms.
\end{options}
\begin{bewareofthedog}
  This will fail miserably together with local lists if an acronym appears in
  more than one list.  This \emph{might} be resolved on day.
\end{bewareofthedog}

\subsection{Bookmarks}\label{sec:bookmarks}
Since bookmarks (which are created by the \pkg{hyperref} or the \pkg{bookmark}
packages~\cite{pkg:bookmark}) can only contain simple text \acro\ simplifies
the output of the acronym commands when they appear in a bookmark.  Although
the output can be modified with a dedicated template-mechanism there is no
user interface at the moment.  Contact me at \issues\ if you need it.

Acronyms have the property \property{pdfstring}:
\begin{properties}
  %% pdfstring
  \propval{pdfstring}{pdfstring}\Default={short}
    Used as \acs{pdf} string replacement for the short form in bookmarks when
    used together with the \pkg{hyperref}~\cite{pkg:hyperref} or the
    \pkg{bookmark} package~\cite{pkg:bookmark}.
\end{properties}

This is for acronyms like
\begin{sourcecode}
  \DeclareAcronym{pdf}{
    short = pdf ,
    long = Portable Document Format ,
    short-format = \scshape ,
    pdfstring = PDF
  }
\end{sourcecode}
where the bookmark would write \enquote{pdf} instead of \enquote{\textsc{pdf}}
if the property where not set.

\subsection{\acs*{pdf} comments}\label{sec:pdf-comments}
Some people like see comments in the \ac{pdf} when they're hovering with the
mouse over the short form of an acronym.  This can be achieved.
\begin{options}
  \keybool{use}\Module{pdfcomments}\Default{false}
    This enables the creation of \ac{pdf} comments.
  \keyval{cmd}{code}\Module{pdfcomments}\Default{\cs*{pdftooltip}\Marg{\#1}\Marg{\#2}}
    Chooses the command for actually creating the comment.  You must refer to
    the printed output in the \ac{pdf} with \code{\#1} and to the comment with
    \code{\#2}.  The default command \cs*{pdftooltip} is provided by the
    package \pkg{pdfcomment}~\cite{pkg:pdfcomment}.  You must load it in order
    to use it.
\end{options}
Only acronyms where the corresponding property has been set will get comments:  
\begin{properties}
  %% pdfcomment
  \propval{pdfcomment}{text}
    Sets a tooltip description for an acronym.
\end{properties}

\subsection{Accessibility support}\label{sec:access-supp}
\acro\ supports the \pkg{accsupp} package~\cite{pkg:accsupp} when you
\emph{also load \pkg{hyperref}}.  Then \acro\ uses
\begin{sourcecode}
  \BeginAccSupp{ method = pdfstringdef ,  ActualText = {PDF} }
    \textsc{pdf}%
  \EndAccSupp{}%
\end{sourcecode}
for an acronym defined like this:
\begin{sourcecode}
  \DeclareAcronym{pdf}{
    short = pdf ,
    long = Portable Document Format ,
    short-format = \scshape ,
    pdfstring = PDF ,
    short-acc = PDF
  }
\end{sourcecode}
Without accessibility support when a string like \enquote{\textsc{pdf}} is
copied from the \ac{pdf} and pasted you get \enquote{pdf}.  If you don't care
about that simply don't load \pkg{accsupp} and ignore this section.

You have a few options to be able to manipulate what \acro\ does here but I
recommend to stay with the default settings:
\begin{options}
  \keybool{use}\Module{accsupp}\Default{true}
    When this is true and the package \pkg{accsupp} is loaded then
    accessibility support is used.
   \keyval{options}{text}\Module{accsupp}\Default
    Additional option to be passed to \cs{BeginAccSupp}.  See the
    \pkg{accsupp} manual for possible settings.
  \keyval{method}{method}\Module{accsupp}\Default{pdfstringdef}
    The method used by \cs{BeginAccSupp}.  See the
    \pkg{accsupp} manual for possible values.
\end{options}

The \enquote{ActualText} that is used by \acro\ always defaults to the values
of the acronym properties themselves.  You can choose these values
individually by setting the corresponding properties:
\begin{properties}
  %% short-acc
  \propval{short-acc}{text}\Default={short}
    Sets the \code{ActualText} property as presented by the \pkg{accsupp}
    package for the short form of the acronym.
  %% long-acc
  \propval{long-acc}{text}\Default={long}
    Sets the \code{ActualText} property as presented by the \pkg{accsupp}
    package for the long form of the acronym.
  %% alt-acc
  \propval{alt-acc}{text}\Default={alt}
    Sets the \code{ActualText} property as presented by the \pkg{accsupp}
    package for the alternative short form of the acronym.
  %% foreign-acc
  \propval{foreign-acc}{text}\Default={foreign}
    Sets the \code{ActualText} property as presented by the \pkg{accsupp}
    package for the foreign form of the acronym.
  %% extra-acc
  \propval{extra-acc}{text}\Default={extra}
    Sets the \code{ActualText} property as presented by the \pkg{accsupp}
    package for the extra information of the acronym.
  %% single-acc
  \propval{single-acc}{text}\Default={long-acc}
    Sets the \code{ActualText} property as presented by the \pkg{accsupp}
    package for a single appearance of the acronym.
  %% list-acc
  \propval{list-acc}{text}\Default={list}
    Sets the \code{ActualText} property as presented by the \pkg{accsupp}
    package for the appearance in the list of acronyms.
\end{properties}

Extra care has to be taken for plural forms as these can not be picked up
automatically right now.  You have to explicitly set them for the
accessibility support, too:
\begin{sourcecode}
  \DeclareAcronym{ufo}{
    short = UFO ,
    long = unidentified flying object ,
    foreign = unbekanntes Flugobjekt ,
    foreign-plural-form = unbekannte Flugobjekte ,
    foreign-acc-plural-form = unbekannte Flugobjekte ,
    foreign-babel = ngerman ,
    long-indefinite = an
  }
\end{sourcecode}

\section{Localisation}\label{sec:localisation}
There are places when \acro\ uses text strings which depend on the language of
the document.  In order to recognize the language from \pkg{babel} of
\pkg{polyglossia} and print the strings in the correct language \acro\ uses
the \pkg{translations}~\cite{pkg:translations}.

If the language is detected incorrectly or you want \acro\ to use another
language than it detects you can use the following option:
\begin{options}
  \keychoice{language}{auto,\meta{language}}\Default{auto}
    The default setting \code{auto} lets \acro\ detect the language setting
    automatically. Valid choices are all language names known to the package
    \pkg{translations}. Mostly just type your language and it should work.
\end{options}
\acro\ only provides support for a handful of languages.  You can easily teach
\acro\ your language -- see section~\vref{sec:translations} -- if it isn't
supported, yet\footnote{If you like you can always open an issue at \issues\
  and provide your translations so I can add them to \acro.}.
\begin{commands}
  \expandable\command{acrotranslate}[\marg{key}]
    This command fetches the translation of \meta{key} for the current
    language.  It is meant for usage in template definitions.
\end{commands}

Available keywords and their English, French, and German translations are
shown in table~\vref{tab:translations}.

\begin{table}
  \centering
  \begin{tabular}{llll}
    \toprule
      \bfseries Key & \bfseries English & \bfseries French & \bfseries German \\
    \midrule
      \translationtable
    \bottomrule
  \end{tabular}
  \caption{Available translation keywords.}
  \label{tab:translations}
\end{table}

\section{Patches}\label{sec:patches}
In several situations it can lead to wrong results if \acro\ marks an acronym
as used too early or at all.  This is why it is possible to disable the
mechanism which is responsible:
\begin{commands}
  \command{acswitchoff}
    This disables the mechanism which marks acronyms as used. After this
    command every acronym command like \cs{ac} acts like its starred version.
  \command{acswitchon}
    This command enables the mechanism again.
\end{commands}

In certain circumstances \acro\ uses these commands itself. For example it is
often preferable that acronyms are not counted as used in floats, the table of
contents or the lists of figures and tables.  This is why \acro\ turns the
mechanism off in these places.

Certain table environments typeset their contents twice for measurement
purposes.  \acro\ tries to disable the usage mechanism during these phases.
The same is true for single line captions from the \pkg{caption} package.

All these patches can be turned off:
\begin{options}
  \keybool{floats}\Module{patch}\Default{true}
    En-/disable the \option{floats} patch.
  \keybool{lists}\Module{patch}\Default{true}
    En-/disable the \option{lists} patch for the table of contents, the list
    of figures and the list of tables.
  \keybool{tabularx}\Module{patch}\Default{true}
    En-/disable the \option{tabularx} patch.
  \keybool{ltxtable}\Module{patch}\Default{true}
    En-/disable the \option{ltxtable} patch.
  \keybool{tabu}\Module{patch}\Default{true}
    En-/disable the \option{tabu} patch.
  \keybool{caption}\Module{patch}\Default{true}
    En-/disable the \option{caption} patch.
\end{options}

\part{Extending \acro}\label{part:extending-acro}
\section{Background}\label{sec:background}
\subsection{Templates}
One of the core ideas of \acro\ version~3.0 is the use of \emph{templates}
which manage how different how anything is printed, from the output of \cs{ac}
and friends to the list of acronyms.  \acro\ uses three types of templates:
\begin{description}
  \item[acronym] These templates can be used to define \emph{acronym
      commands}, see section~\vref{sec:own-acronym-commands}.
  \item[list] These templates are used by the \cs{printacronyms} command.
  \item[heading]  These templates only make sense if a \emph{list} template
    uses \cs{acroheading}.  This command makes use of them.
\end{description}

How these templates are defined, which are available from the start and how
they are used is explained in section~\vref{sec:templates}.

\subsection{Objects}
\acro\ uses certain kinds of objects in some of its commands.  It is possible
to defines own such objects:
\begin{description}
  \item[articles] Per default only the \enquote{indefinite} article is
    defined. But it is possible to define and add other articles to \acro.
    This is explained in section~\vref{sec:other-articles}.
  \item[endings] Per default only the ending \enquote{plural} is defined. But
    it is possible to define and add other endings to \acro.  This is
    explained in section~\vref{sec:other-endings}.
  \item[properties] You have already learned about properties.  It is possible
    to define and add further acronym properties to \acro.  This is explained
    in section~\vref{sec:new-properties}.
  \item[translations] \acro\ uses localisation strings at a number of places.
    It is possible to change these strings and add further strings.  This is
    explained in section~\vref{sec:localisation}.
\end{description}

\section{Templates}\label{sec:templates}
\subsection{Pre-defined templates}\label{sec:pre-defin-templ}
\subsubsection{Acronym templates}
\begin{templates}
  \tmpl{alt}
    Display the alternative form of an acronym.
  \tmpl{first}
    This is a \emph{pseudo} template which always displays what is
    set through the option \option{first-style} or the property
    \property{first-style}.
  \tmpl{footnote}
    A template for the first appearance where the long form is printed in a
    footnote.
  \tmpl{long}
    Display the long form of an acronym.
  \tmpl{long-short}
    A template for the first appearance where the long form is printed and the
    short form follows in parentheses.
  \tmpl{short}
    Display the short form of an acronym.
  \tmpl{short-long}
    A template for the first appearance where the short form is printed and the
    long form follows in parentheses.
  \tmpl{single}
    A template which is used when the property \property{single} has been set
    \emph{and} the option \option{single} has been set \emph{and} if the
    acronym is only used a single time.
  \tmpl{show}
    A template which writes all properties of an acronym into the log file.
\end{templates}

\subsubsection{List templates}
\begin{templates}
  \tmpl{description}
    The default list style which places the short form in the item of a
    \env*{description} environment and adds the all the rest as description of
    the item.
  \tmpl{lof}
    A style which mimicks the list of figures.  This style does not support
    page ranges.
  \tmpl{longtable}
    A style that uses a \env*{longtable} environment for building the list.
    This needs the \pkg{longtable} package~\cite{pkg:longtable} loaded.
  \tmpl{longtabu}
    A style that uses a \env*{longtabu} environment for building the
    list. This needs the \pkg{longtable} package and the \pkg{tabu}
    package~\cite{pkg:tabu} loaded\footnote{Please note that this package
      currently is un-maintained and has a number of open bugs. For further
      information refer to
      \url{https://github.com/tabu-issues-for-future-maintainer/tabu}}.
  \tmpl{supertabular}
    A\sinceversion{3.2} style that uses a \env*{supertabular} environment for
    building the list.  This needs the \pkg{supertabular}
    package~\cite{pkg:supertabular} loaded.
  \tmpl{tabular}
    A style that uses a \env*{tabular} environment for building the list.
    Since a \env*{tabular} cannot break across pages this is only suited for
    short lists.
  \tmpl{toc}
    A style which mimicks the table of contents.  This style does not support
    page ranges.
\end{templates}

\subsubsection{Heading templates}
\begin{templates}
  \tmpl{addchap}
    Only defined in a \KOMAScript\ class and if \cs*{chapter} is defined. Uses
    \cs*{addchap} for the heading.
  \tmpl{addsec}
    Only defined in a \KOMAScript\ class. Uses \cs*{addsec} for the heading.
  \tmpl{chapter}
    Only defined if \cs*{chapter} is defined. Uses \cs*{chapter} for the heading.
  \tmpl{chapter*}
    Uses \cs*{chapter}\sarg\ for the heading.
  \tmpl{none}
    Displays nothing.
  \tmpl{section}
    Uses \cs*{section} for the heading.
  \tmpl{section*}
    Uses \cs*{section}\sarg\ for the heading.
\end{templates}

\subsection{Defining new templates}\label{sec:defin-new-templ}
For the definition of templates these commends are available:
\begin{commands}
  \command{NewAcroTemplate}[\oarg{type}\marg{name}\marg{code}]
    This defines a template of type \meta{type} with the name \meta{name}
    which inserts \meta{code} when used.  A template of type \meta{type} with
    name \meta{name} must not exist.  The default type is \code{acronym}.
  \command{RenewAcroTemplate}[\oarg{type}\marg{name}\marg{code}]
    This re-defines a template of type \meta{type} woth the name \meta{name}
    which inserts \meta{code} when used.  A template of type \meta{type} with
    name \meta{name} must exist.  The default type is \code{acronym}.
  \command{SetupAcroTemplate}[\oarg{type}\marg{name}\marg{code}]
    Adds\sinceversion{3.2} \meta{code} to the beginning of the template
    \meta{name} of type \meta{type}.  The default type is \code{acronym}.
  \command{SetupNextAcroTemplate}[\oarg{type}\marg{name}\marg{code}]
    Adds\sinceversion{3.2} \meta{code} to the beginning of \emph{the next use}
    of the template \meta{name} of type \meta{type}.  The default type is
    \code{acronym}.
  \expandable\command{AcroTemplateType}
    Within a template this expands to the \meta{type} of the current
    template.
  \expandable\command{AcroTemplateName}
    Within a template this expands to the \meta{name} of the current
    template.
\end{commands}

How to use these commands is best explained by examples of how the existing
templates have been defined.  The following sections will show several
examples for their usage.

\subsection{Commands to be used in template definitions}

\acro\ provides and uses a large number of commands that are meant to be used
in template definitions and that often are useless or will raise errors if
used outside.  Depending on their purpose the commands can be used in
different types of templates or only in certain types of templates.

In the descriptions below a \expandablesymbol\ indicates a fully expandable
command when used in an \cs*{edef}, \cs*{write} or in \cs*{expanded}.

A \TF\ always refers to a \meta{true} and \meta{false} branch and indicates
that \emph{three} commands exist: one exactly as described, one with only the
\code{\textcolor{cs}{T}} and the \meta{true} branch, and one with only the
\code{\textcolor{cs}{F}} and the \meta{false} branch.  So \cs{acroif\TF} means
there is \cs*{acroifTF}, \cs*{acroifT}, and \cs*{acroifF}, where \cs*{acroifT}
and \cs*{acroifF} each have an argument less than \cs*{acroifTF}.

\subsubsection{Commands for common uses}
\begin{commands}
  \expandable\command{acrolistname}
    Expands to whatever is currently set with
    \module{list}\code{/}\option{name}.
  \command{acrowrite}[\marg{property}]
    Prints the property \meta{property} of the current acronym.  Depending on
    the circumstances this prints the property together with an article or an
    ending either in uppercase or lowercase form.  Default is the lowercase
    form without ending or article. The actual outcome is determined by
    switches which are explained in section~\vref{sec:own-acronym-commands}.
  \command{acroformat}[\marg{type}\marg{text}]
    This formats \meta{text} according to \meta{type} where \meta{type} has
    either been set as property or as option from the \module{format} module.
    Valid values are \code{short}, \code{long}, \code{alt}, \code{extra},
    \code{foreign}, \code{list}, and \code{first-long}.
  \command{acroshow}[\marg{property}]
    For debugging puposes: writes the property \meta{property} of the current
    acronym to the log file.
  \command{acroif\TF}[\marg{property}\marg{true}\marg{false}]
    Checks if the property \meta{property} has been set for the current
    acronym and either leaves \meta{true} or \meta{false} in the input stream.
  \command{acroifboolean\TF}[\marg{property}\marg{true}\marg{false}]
    \sinceversion{3.1}Returns \meta{true} if the boolean property
    \meta{property} has been set to \code{true} and \meta{false} otherwise.
  \command{acroifall\TF}[\marg{properties}\marg{true}\marg{false}]
    Checks if all properties in the csv list \meta{properties} have been set
    for the current acronym and either leaves \meta{true} or \meta{false} in
    the input stream.
  \command{acroifany\TF}[\marg{properties}\marg{true}\marg{false}]
    Checks if any of the properties in the csv list \meta{properties} has been
    set for the current acronym and either leaves \meta{true} or \meta{false}
    in the input stream.
  \command{acroiftag\TF}[\marg{tag}\marg{true}\marg{false}]
    Checks if the current acronym has been given the tag \meta{tag} and either
    leaves \meta{true} or \meta{false} in the input stream.
  \command{AcroPropertiesMap}[\marg{code}]
    Maps over all defined acronym properties.  Within \meta{code} you can
    refer to the current property with \code{\#1}.
\end{commands}

\subsubsection{Commands for usage in acronym templates}
\begin{commands}
  \expandable\command{acroifused\TF}[\marg{true}\marg{false}]
    Checks if the current acronym has been used before and either leaves
    \meta{true} or \meta{false} in the input stream.
  \command{acroiffirst\TF}[\marg{true}\marg{false}]
    Checks if the current usage of the current acronym is the first time and
    either leaves \meta{true} or \meta{false} in the input stream.
  \command{acroifsingle\TF}[\marg{true}\marg{false}]
    Checks if the current acronym is used a single time and either leaves
    \meta{true} or \meta{false} in the input stream.
  \command{acrogroupcite}
\end{commands}

\subsubsection{Commands for usage in list templates}
\begin{commands}
  \expandable\command{acroifchapter\TF}[\marg{true}\marg{false}]
    This just check if \cs*{chapter} is defined.  Used in the \template{toc}
    template.
  \expandable\command{acroifpages\TF}[\marg{true}\marg{false}]
    This is \meta{true} if the option \module{pages}\code{/}\option{display}
    is set, \emph{and} the current acronym is not single, \emph{and} has at
    least one page number.  \meta{false} otherwise.
  \command{acropages}[\marg{first}\marg{range}]
    If \cs{acroifpages\TF} would be \meta{false} this would do nothing.
    Otherwise, if \module{pages}\code{/}\option{display} is \code{first} it
    prints the first page number, preceded by \meta{first} if
    \module{pages}\code{/}\option{name} is true.  If
    \module{pages}\code{/}\option{display} is \code{all} it prints the page
    range, preceded by \meta{range} if \module{pages}\code{/}\option{name} is
    true.
  \command{acronopagerange}
    This disables page ranges.  Used in the \template{toc} and \template{lof}
    templates.
  \command{acroneedpages}
    \sinceversion{3.4}This enables the page number displayed.  Used in the
    \template{toc} and \template{lof} templates. 
  \command{acropagefill}
    If \cs{acroifpages\TF} would be \meta{false} this would do
    nothing. Otherwise it prints whatever is set by
    \module{pages}\code{/}\option{fill}.
  \command{acronymsmap}[\marg{code}]
    Maps over the acronyms in order of appearance in the list.  Which acronyms
    these are depends on settings.  They might only have certain tags, be ones
    local to barriers, \dots \par
    Within \meta{code} \code{\#1} refers to the current \ac{id} of the
    acronym.  Also \cs{AcronymID} expands to the current \ac{id}.  The latter
    is important for all the commands that check or print properties of
    acronyms.
  \command{acronymsmap\TF}[\marg{code}\marg{true}\marg{false}]
    This does the same as \cs{acronymsmap} and also leaves \meta{true} in the
    input stream if the list is not empty and \meta{false} otherwise.  This is
    useful to trigger a rerun warning.
  \command{AcronymTable}
    This is an empty token list at the beginning of a list template.
  \command{AcroAddRow}[\marg{code}]
    Adds \meta{code} to the right of \cs{AcronymTable} and ensures that
    \cs{AcronymID} has the correct global definition for this code.  With this
    the code for the \template{tabular} template and other table templates can
    be built in a comfortable way.
  \command{AcroNeedPackage}[\marg{package}]
    Checks if the package \meta{package} is loaded and throws an error
    otherwise.
  \command{AcroRerun}
    Triggers \acro\ to throw an \enquote{empty list} rerun warning.
\end{commands}

\subsection{New acronym templates}\label{sec:new-acronym-templ}

Some templates are quite short and self-explaining:
\begin{sourcecode}
  \NewAcroTemplate{short}{\acrowrite{short}}
\end{sourcecode}

Some are a little bit more elaborate:
\begin{sourcecode}
  \NewAcroTemplate{alt}{%
    \acroifTF{alt}
      {\acrowrite{alt}}
      {\acrowrite{short}}%
  }
\end{sourcecode}

And some templates need to do a lot more:
\begin{sourcecode}
  \NewAcroTemplate{long-short}{%
    \acroiffirstTF{%
      \acrowrite{long}%
      \acspace(%
        \acroifT{foreign}{\acrowrite{foreign}, }%
        \acrowrite{short}%
        \acroifT{alt}{ \acrotranslate{or} \acrowrite{alt}}%
        \acrogroupcite
      )%
    }%
    {\acrowrite{short}}%
  }
\end{sourcecode}

\subsection{New list templates}\label{sec:new-list-templates}
This section shows the definition of three templates: \template{description},
\template{tabular}, and \template{toc}.

First the \template{description} template:
\begin{sourcecode}
  \NewAcroTemplate[list]{description}{%
    \acroheading
    \acropreamble
    \begin{description}
      \acronymsmapF{%
        \item[\acrowrite{short}\acroifT{alt}{/\acrowrite{alt}}]
          \acrowrite{list}%
          \acroifanyT{foreign,extra}{ (}%
          \acroifT{foreign}{\acrowrite{foreign}\acroifT{extra}{, }}%
          \acroifT{extra}{\acrowrite{extra}}%
          \acroifanyT{foreign,extra}{)}%
          \acropagefill
          \acropages
            {\acrotranslate{page}\nobreakspace}
            {\acrotranslate{pages}\nobreakspace}%
        }
        {\item\AcroRerun}
    \end{description}
  }
\end{sourcecode}

The following shows how to define templates using some kind of table
environment.  Special care is necessary due to the way \LaTeX\ tables work:
first the table body is built and only then the table itself is printed:
\begin{sourcecode}
  \NewAcroTemplate[list]{tabular}{%
    \AcroNeedPackage{array}%
    \acronymsmapF{%
      \AcroAddRow{
        \acrowrite{short}%
        \acroifT{alt}{/\acrowrite{alt}}
        &
        \acrowrite{list}%
        \acroifanyT{foreign,extra}{ (}%
        \acroifT{foreign}{\acrowrite{foreign}\acroifT{extra}{, }}%
        \acroifT{extra}{\acrowrite{extra}}%
        \acroifanyT{foreign,extra}{)}%
        \acropagefill
        \acropages
          {\acrotranslate{page}\nobreakspace}
          {\acrotranslate{pages}\nobreakspace}%
        \tabularnewline
      }%
    }
    {\AcroRerun}%
    \acroheading
    \acropreamble
    \par\noindent
    \begin{tabular}{>{\bfseries}lp{.7\linewidth}}
      \AcronymTable
    \end{tabular}
  }
\end{sourcecode}

\begin{sourcecode}
  \NewAcroTemplate[list]{toc}{%
    \acroheading
    \acropreamble
    \acronopagerange
    \acronymsmapF{%
      \contentsline{\acroifchapterTF{chapter}{section}}
        {\acrowrite{short}\acroifT{alt}{/\acrowrite{alt}}}
        {}{}%
      \contentsline{\acroifchapterF{sub}section}
        {
          \acrowrite{list}%
          \acroifT{foreign}{\acrowrite{foreign}\acroifT{extra}{, }}%
          \acroifT{extra}{\acrowrite{extra}}%
          \acroifanyT{foreign,extra}{)}%
        }
        {\acropages{}{}}
        {}%
    }
    {\AcroRerun}
  }
\end{sourcecode}

\subsection{New heading templates}\label{sec:new-head-templ}
Let's take a look at the two templates \template{section} and
\template{section*} which should give you enough information to build your
own:
\begin{sourcecode}
  \NewAcroTemplate[heading]{section} {\section {\acrolistname}}
  \NewAcroTemplate[heading]{section*}{\section*{\acrolistname}}
\end{sourcecode}

\section{Endings}\label{sec:endings}
Referring to section~\vref{sec:other-endings} this section explains how to
define and use additional endings.
\begin{commands}
  \command{DeclareAcroEnding}[\marg{name}\marg{short default}\marg{long default}]
    This command can be used to define properties and options analoguous to
    the plural endings which have been defined this way:
\end{commands}
\begin{sourcecode}
  \DeclareAcroEnding{plural}{s}{s}
\end{sourcecode}
In general \cs{DeclareAcroEnding}\marg{foo}\marg{x}\marg{y} defines these
options
\begin{options}
  \keyval*{short-\meta{foo}-ending}{value}\Default*{\meta{x}}
  \keyval*{long-\meta{foo}-ending}{value}\Default*{\meta{y}}
\end{options}
and these properties
\begin{properties}
  \propval*{short-\meta{foo}}{value}\Default*{\meta{x}}
  \propval*{short-\meta{foo}-form}{value}\Default*
  \propval*{alt-\meta{foo}}{value}\Default*{\meta{x}}
  \propval*{alt-\meta{foo}-form}{value}\Default*
  \propval*{long-\meta{foo}}{value}\Default*{\meta{y}}
  \propval*{long-\meta{foo}-form}{value}\Default*
  \propval*{foreign-\meta{foo}}{value}\Default*{\meta{y}}
  \propval*{foreign-\meta{foo}-form}{value}\Default*
  \propval*{single-\meta{foo}}{value}\Default*{\meta{y}}
  \propval*{single-\meta{foo}-form}{value}\Default*
  \propval*{extra-\meta{foo}}{value}\Default*{\meta{y}}
  \propval*{extra-\meta{foo}-form}{value}\Default*
\end{properties}
In addition another command is defined which is meant to be used in
template definitions.
\begin{commands}
  \command*{acro\meta{foo}}
    This command tells the template that the ending \meta{foo} should be
    used.
\end{commands}
Section~\vref{sec:own-acronym-commands} has an example of how this can be used
to define a possessive ending and commands that make use of them like this:
\begin{example}[side-by-side]
  \acfg{MP}
\end{example}

\section{Articles}\label{sec:articles}
Referring to section~\vref{sec:other-articles} this section explains how to
define and use additional articles.
\begin{commands}
  \command{DeclareAcroArticle}[\marg{name}\marg{default}]
    This command can be used to define properties and options analoguous to
    the indefinite article which have been defined this way:
\end{commands}
\begin{sourcecode}
  \DeclareAcroArticle{indefinite}{a}
\end{sourcecode}
In general \cs{DeclareAcroArticle}\marg{foo}\marg{x} defines the
option
\begin{options}
  \keyval*{\meta{foo}}{value}\Default*{\meta{x}}
\end{options}
and these properties
\begin{properties}
  \propval*{short-\meta{foo}}{value}\Default*{\meta{x}}
  \propval*{alt-\meta{foo}}{value}\Default*{\meta{x}}
  \propval*{long-\meta{foo}}{value}\Default*{\meta{x}}
  \propval*{foreign-\meta{foo}}{value}\Default*{\meta{x}}
  \propval*{single-\meta{foo}}{value}\Default*{\meta{x}}
  \propval*{extra-\meta{foo}}{value}\Default*{\meta{x}}
\end{properties}
In addition another command is defined which is meant to be used in
template definitions.
\begin{commands}
  \command*{acro\meta{foo}}
    This command tells the template that the article \meta{foo} should be
    used.
\end{commands}
Section~\vref{sec:own-acronym-commands} has examples of how this can be used
to define definite articles and commands that make use of them like this:
\begin{example}
  \dacs{hadopi} \par
  \dacl{hadopi}
\end{example}

\section{Translations}\label{sec:translations}
For adding additional keywords, or for adding translations to existing
keywords, or for changing existing translations \acro\ uses this command:
\begin{commands}
  \command{DeclareAcroTranslation}[\marg{key}\marg{language=translation list}]
    With this command new translations keywords can be added and translations
    for existing keywords can be changed.
  \command{AddAcroTranslations}[\marg{key}\marg{language=translation list}]
    Basically the same but this time per language rather than per keyword.
\end{commands}
As an example this is how \acro\ declares translations for the \code{pages}
keyword:
\begin{sourcecode}
  \DeclareAcroTranslation{pages}{
    Fallback   = pp\abbrdot ,
    English    = pp\abbrdot ,
    French     = pp\abbrdot ,
    German     = S\abbrdot ,
    Portuguese = pp\abbrdot
  }
\end{sourcecode}
Translations for a language could be added this way\footnote{\acro\ already
  has the translations for Italian.}:
\begin{sourcecode}
  \AddAcroTranslations{Italian}{
    list-name = Acronimi ,
    page      = p\abbrdot ,
    pages     = pp\abbrdot ,
    sequens   = s\abbrdot ,
    sequentes = ss\abbrdot ,
    also      = anche ,
    and       = e ,
    or        = o
  }
\end{sourcecode}
The existing keywords had been shown in table~\vref{tab:translations}.

\section{Properties}\label{sec:new-properties}
As you know from section~\ref{sec:decl-acronyms-other} \acro\ comes with quite
a number of predefined properties which control various aspects of
acronyms. However, there are cases when additional properties would be nice to
have and to use. It can be done with the following command:
\begin{commands}
  \command{DeclareAcroProperty}[\sarg\arg{?}\arg{!}\arg{|}\arg{>}\marg{name}]
    This defines the new property \meta{name}.  The command has five optional
    arguments most of which you probably never need. \par
    The optional star \sym{*} ensures that each acronym gets a \emph{unique}
    value for the property. \par
    The optional question mark \sym{?} creates a \emph{boolean} property.
    That is a property that only can get the values \code{true} or
    \code{false} and when it is used without value (not an empty value!) then
    \code{true} is assumed. \par
    The optional exclamation mark \sym{!} creates a \emph{mandatory}
    property. An error is raised if an acronym does not set it. \par
    The optional pipe \sym{|} creates a \emph{static} property which means
    its value is written to an auxiliary file and read in again at begin
    document.  Once set the value is the same throughout the document. \par
    The optional greater as symbol\sinceversion{3.2} \sym{>} creates a
    \emph{display} property.  This additionally defines the two boolean
    options \meta{name}\code{/}\option{display} and
    \module{list}\code{/}\meta{name}\code{/}\option{display}, both initially
    set to true.  If these options are set to false the acronym commands or
    the list act as if the property \meta{name} has not been set.  The
    \property{foreign} property is an example.
  \command{DeclareAcroPropertyAlias}[\sarg\arg{?}\arg{!}\arg{|}\arg{>}\marg{name1}\marg{name2}]
    This newly declares property \meta{name1} and makes it an alias of
    property \meta{name2}.  This means that \meta{name1} gets the same value
    that \meta{name2} has unless it is set explicitly.  Property \meta{name2}
    must exist.
  \command{MakeAcroPropertyAlias}[\marg{name1}\marg{name2}]
    This makes property \meta{name1} and makes it an alias of property
    \meta{name2}. Both properties must exist.
\end{commands}
Exmaples for defining and using new properties are shown in
section~\ref{sec:examples}, \ac*{eg}, examples~\ref{example:texsx-505891}
or~\ref{example:texsx-507726}.

\section{Own acronym commands}\label{sec:own-acronym-commands}
\subsection{Background}
You can define own acronym commands or redefine the existing ones with
commands similar to \cs*{NewDocumentCommand} from the \pkg{xparse}
package~\cite{pkg:xparse}.
\begin{commands}
  \command{NewAcroCommand}[\marg{command}\marg{arg. spec.}\marg{code}]
    This creates the new command \meta{command} with the argument
    specification\footnote{in the sense of an \pkg{xparse}
      command.} \code{so\meta{arg. spec.}} and replacement text \meta{code}.
    There are significant differences to \cs*{NewDocumentCommand}: the new
    command always has two additional arguments: an optional star and an
    optional argument for options.  You can ignore this fact in your
    definition, though.  However, the command \emph{must} at least have one
    argument \emph{and} the first argument \emph{must} refer to the \ac{id}.
    Everything else is up to you. \par
    The new command has the suiting framework to recognize trailing tokens,
    count usage, index, and add a citation if necessary.
  \command{RenewAcroCommand}[\marg{command}\marg{arg. spec.}\marg{code}]
    Like \cs{NewAcroCommand} but redefines an existing command.
  \command{UseAcroTemplate}[\oarg{type}\marg{name}\oarg{argument number}\meta{arguments}]
    The argument \meta{type} defaults to \code{acronym} and \meta{argument
      number} defaults to \code{1}.  The command must be followed by as many
    mandatory arguments as you specify with \meta{argument number}.  All
    predefined acronym templates use the first argument as \ac{id} so they
    must use one argument.
\end{commands}
Let's see an example. This is the definition of \cs{ac}:
\begin{sourcecode}
  \NewAcroCommand\ac{m}{\UseAcroTemplate{first}{#1}}
\end{sourcecode}
Equivalent definitions would be:
\begin{sourcecode}
  \NewAcroCommand\ac{m}{\UseAcroTemplate[acroynm]{first}{#1}}
  \NewAcroCommand\ac{m}{\UseAcroTemplate[acroynm]{first}[1]{#1}}
  \NewAcroCommand\ac{m}{\UseAcroTemplate{first}[1]{#1}}
  \NewAcroCommand\ac{m}{\UseAcroTemplate{first}[2]{#1}{}}
\end{sourcecode}

There are a number of switch commands which determine a certain behavior.
They tell the following template how to interpret certain conditionals and how
to use \cs{acrowrite}.
\begin{commands}
  \command{acrocite}
    Tells \acro\ to output the citation.
  \command{acrodonotuse}
    Tells \acro\ to not count this as usage.
  \command{acroplural}
    Use plural form.
  \command{acroindefinite}
    Use indefinite article
  \command{acroupper}
    Use uppercase form.
  \command{acrofull}
    Use first or full form.
\end{commands}

Here is an example that makes use of them:
\begin{sourcecode}
  \NewAcroCommand\Iacs{m}{%
    \acroupper\acroindefinite\UseAcroTemplate{short}{#1}%
  }
\end{sourcecode}

\subsection{Create commands for possessive endings}
Let's say you want to add an ending for the genitive case.
First you define the appropriate ending:
\begin{sourcecode}
  \DeclareAcroEnding{possessive}{'s}{'s}
\end{sourcecode}
Then you define commands which make use of this ending:
\begin{sourcecode}
  \NewAcroCommand\acg{m}{\acropossessive\UseAcroTemplate{first}{#1}}
  \NewAcroCommand\acsg{m}{\acropossessive\UseAcroTemplate{short}{#1}}
  \NewAcroCommand\aclg{m}{\acropossessive\UseAcroTemplate{long}{#1}}
  \NewAcroCommand\acfg{m}{%
    \acrofull
    \acropossessive
    \UseAcroTemplate{first}{#1}%
  }
  \NewAcroCommand\iacsg{m}{%
    \acroindefinite
    \acropossessive
    \UseAcroTemplate{short}{#1}%
  }
\end{sourcecode}
You maybe also define acronyms with corresponding properties\footnote{Bear
  with me if this is incorrect: English is not my native language.}:
\begin{sourcecode}
  \DeclareAcronym{MP}{
    short = MP ,
    long  = Member of Parliament ,
    plural-form = Members of Parliament ,
    long-possessive-form = Member's of Parliament
  }
\end{sourcecode}
Now you can use it like this:
\begin{example}
  This is the \acg{MP} first day at work after \dots
\end{example}

\section{Own \acro\ style files}
When you want to use your definitions regarding \acro\ repeatedly then it
makes sense to put them in a file which you put somewhere in your local
\LaTeX\ tree.  There are three options:
\begin{enumerate}
  \item Put them in a simple \code{.tex} file in \cs*{input} it.
  \item Put in in a \code{.sty} file and include it with \cs*{usepackage}
    \emph{after} \acro.
  \item Create a style file following this pattern decribed below.
\end{enumerate}
\begin{center}
  \code{acro.style.\meta{name}.code.tex}
\end{center}
This file should start with
\begin{sourcecode}
  \AcroStyle{name}
\end{sourcecode}
and input the file with \cs{acsetup} using the option
\begin{options}
  \keyval{load-style}{name}
    This is more or less the same as if you'd use the package variant but
    naturally ensures that you load it after \acro\ and in the future might
    provide other bells and whistles, too.
\end{options}
The command
\begin{commands}
  \command{AcroStyle}[\sarg\marg{style}\oarg{details}]
    has an optional star which switches to expl3 syntax.  It also has an
    optional argument \meta{details} with the same purpose and usage as
    the one from \cs*{ProvidesPackage}. A typical usage would look like
\end{commands}
\begin{sourcecode}
  \AcroStyle{abbrev}[2020/04/21 abbreviations with acro (CN)]
  \NewDocumentCommand\newabbreviation{mmm}{%
    \DeclareAcronym{#1}{ short = #2 , #3 , class = abbrev , no-index }%
  }
  \NewDocumentCommand\printabbreviations{O{}}{%
    \printacronyms[#1,include=abbrev]%
  }
\end{sourcecode}
  
\clearpage
\appendix
\part{Appendix}\label{part:appendix}

\section{Examples}\label{sec:examples}

\listexamplefiles

\printacronyms[
  preamble = {\label{sec:acronyms}Below all abbreviations are listed which
    have been defined for the manual.} ,
  exclude = {exclude,glossary}
]

\printbibliography

\end{document}
